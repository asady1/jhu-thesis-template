\chapter{Introduction}\label{Sec:Intro}
\begin{figure}[h]
    \centering
        \includegraphics[width=0.33\textwidth]{F1/Cloud}
        \caption{Snapshot of a cloud chamber, circa 1933: the vertical line represents a positron traveling through the cloud chamber, curving due to the magnetic field applied to the chamber.}
        \label{Fig:Intro:Elec}
\end{figure}

In 1869, Johann Wilhelm Hittorf discovered cathode rays\cite{cathode}, suggesting that there may be subatomic particles. By 1911, these were identified as electrons, observed in many different ways, including cloud chambers, as can be seen in Figure \ref{Fig:Intro:Elec}. Soon after, protons were discovered in 1917, followed by neutrons in 1935.

As physicists advanced the study of quantum mechanics, many new particles were discovered. These particles can be broken down into several elementary particles, whose interactions are governed by \textit{The Standard Model} (SM).

\section{The Standard Model}
The Standard Model (SM) describes the interaction of all elementary particles with one formula, shown in Figure \ref{Fig:Intro:SML}. This lengthy Lagrangian can be summarized as a list of particles and mediating forces, represented in Figure \ref{Fig:Intro:SMTable}.
\begin{figure}[ht]
    \centering
        \includegraphics[width=0.75\textwidth]{F1/StandardModelEquation}
        \caption{The Standard Model Lagrangian.}
        \label{Fig:Intro:SML}
\end{figure}
\begin{figure}[h]
    \centering
        \includegraphics[width=\textwidth]{F1/StandardModelTable}
        \caption{Table of Standard Model particles. Image from CERN~\cite{SMTable}.}
        \label{Fig:Intro:SMTable}
\end{figure}

\clearpage
\subsection{Leptons}
A lepton is one two categories of matter particles in the SM. The electron is one of the most well-known leptons. There exist two heavier ``versions" of the electron: the muon ($\mu$) which is $\sim 200$ times as massive as the electron, and the tau ($\tau$) which is $\sim 4000$ times as massive as the electron. Each of these particles is negatively charged, where the charge is equivalent to $-1$ which corresponds to $- 1.602 \times 10^{-19}$ Coulombs. 

For each of these particles, there exists a corresponding neutrino: the electron $\nu_e$, the muon neutrino $\nu_\mu$, and the tau neutrino $\nu_\tau$. The neutrinos as defined by the SM are chargeless and massless, although experiments have proved that at least two out of three neutrinos observed in nature have mass. The electron and its corresponding neutrino are first generation fermions, while the muon and its neutrino are second generation and the tau and its neutrino are third generation. 

A corresponding anti-particle exists for each of these six particles: $\overline{e}$ ($e^+$), $\overline{\mu}$ ($\mu^+$,), $\overline{\tau}$ ($\tau^+$), $\overline{\nu}_e$, $\overline{\nu}_\mu$, and $\overline{\nu}_\tau$. Note that, while neutrinos are chargeless, the anti-particle for the electron, muon, and tau, all have positive (or opposite of their corresponding particle) charge. However, each anti-particle is identical in mass to its corresponding particle.

In addition to charge and mass, particles also have spin, or the intrinsic angular momentum of a particle. All of the above mentioned particles, known as leptons, have spin $\frac{1}{2}$ (or $-\frac{1}{2}$), which means they are fermions. The spin of a particle allows us to define helicity, also known as handedness. This is the sign of the projection of the particle's spin vector onto its momentum vector. A particle is said to be right-handed if the spin and momentum align, and left-handed if the spin and momentum point opposite each other. In the Standard Model, there are differences between left-handed and right-handed particles. For example, we have observed that all neutrinos are left-handed, while all anti-neutrinos are right-handed, which is currently still inexplicable. 

Lastly, leptons have another type of "charge" which are related to interactions mediated by the weak force, called weak hypercharge. Right-handed electrons, muons, and taus have $-2$ hypercharge, while left-handed leptons have $-1$ hypercharge. As with electric charge, the sign is opposite for anti-particles.

\subsection{Quarks}
The nucleus of an atom is made up of protons and neutrons, which are two examples of hadrons, or composite particles made up of elementary particles called quarks. There are six quarks: \textit{up} ($u$) and \textit{down} ($d$) (first generation), \textit{strange} ($s$) and \textit{charm} ($c$) (second generation), and \textit{bottom} ($b$) and \textit{top} ($t$) (third generation).

The $u$, $c$ and $t$ quarks have charge $+\frac{2}{3}$ and the $d$, $s$ and $b$ quarks have charge $-\frac{1}{3}$. Quarks also have each a different mass. For every quark there is an anti-quark ($\overline{u}$, $\overline{d}$, $\overline{s}$, $\overline{c}$, $\overline{b}$ and $\overline{t}$) with an opposite charge. Quarks are also fermions, and therefore have a spin of $\frac{1}{2}$ or $-\frac{1}{2}$.

In addition to charge (by which we mean electric charge), mass, and spin, quarks also have a second kind of charge, known as color. While electric charge allows particles to interact through the electromagnetic force, color charge allows particles to interact through a different type of force, called the strong force. Quarks can be red, green, or blue colored, and anti-quarks can be anti-red, anti-green, or anti-blue colored. In nature, we only see color neutral particles, which is why quarks are only observed as combinations forming composite particles. These composite particles made up of quarks are color neutral because they either contain one quark of each color, or a quark of a particular color and a quark of the corresponding anti-color. For example, the proton is made up of one $u$ and two $d$ quarks, so one of these must be red, one must be green, and one must be blue.

Lastly, quarks also have weak hypercharge. Right-handed up-type quarks (u, c, t) have a weak hypercharge of $\frac{4}{3}$, while right-handed down-type quarks (d, s, b) have a weak hypercharge of $-\frac{2}{3}$. Left-handed quarks have a weak hypercharge of $\frac{1}{3}$. As with leptons, the sign of these values is reversed for anti-particles.

\subsection{Feynman Diagrams}

%Consider the most basic of particle interactions: two electrons are shot towards each other. Since they are both negatively charged, they repel each other. If we consider that the system starts in some state $i$ and ends in some state $f$, the probability of the transition $i\rightarrow j$ happening is related to the infinite sum:
%\begin{equation}
%    S_{fi} = \sum_{n=0}^{\infty}S^n
%\end{equation}
%The sum is over orders of perturbation\footnote{Quantum Mechanics uses perturbation theory to move from a simple approximate solution to a more precise one.}, where the first order $n=1$ state is the case where the electrons do not interact at all. The next (and first interesting) term of S is:
%\begin{equation}
%    S^{\{2\}} = \pm^{\{n\}}\bar\psi(x)ie\gamma^\mu\psi(x)\bar\psi(x^\prime)ie\gamma^\nu\psi(x^\prime)\int\frac{d^4k}{(2\pi)^4}\frac{-g_{\mu\nu}}{k^2-i0}e^{-ik(x-x^\prime)}
%\end{equation}
%where the integral is over all the possible momenta of the incoming and outgoing electrons. This is just the first non-trivial term of the sum. Fortunately it is the dominant term. Unfortunately, a clear description of the process is hidden by the opaqueness of the mathematics. Enter the \textit{Feynman diagram}. Each term in the expression for $S^{\{2\}}$ can be encoded in as a pictoral element of a Feynman Diagram (as shown in Figure \ref{Fig:Intro:Feynman0}). 
%\begin{figure}[h]
 %   \centering
%        \includegraphics[width=0.49\textwidth]{F1/Fee}
%        \caption{$S^{\{2\}}$ represented as a Feynman Diagram. Each term in the equation corresponds to either a propagator (a line) or a verted (a ``corner''). The terms \underline{$A_\mu(x)A_\nu(x^\prime)$} is the integral term.}
%        \label{Fig:Intro:Feynman0}
%\end{figure}
%Now that we are armed with the Feynman Diagram, each force can be thought of as some mediator particles with some rules governing which vertexes may be constructed. We will allow the Feynman Diagrams to obfuscate the computations, and focus on a qualitative description of these processes.

The Standard Model describes how matter particles, described above, interact with each other through the exchange of force mediating particles. Each term in the Lagrangian in Figure~\ref{Fig:Intro:SML} represents different types of interactions between particles. Let us consider a simple scenario of electron electron interaction, which happens through the exchange of the electromagnetic mediating particle known as the photon. While the mathematical term associated with this process is important, we can still learn a lot by examining the pictorial representation of this term, as can be seen in Figure \ref{Fig:Intro:Vertex1}. 
\begin{figure}[h]
    \centering
        \includegraphics[width=0.33\textwidth]{F1/Vertex1}
        \caption{Visual representation of the interaction between two electrons (straight lines) and a photon (wavy line).}
        \label{Fig:Intro:Vertex1}
\end{figure}
This vertex can be arranged and combined with other vertices in a variety of ways to represent different interactions between electrons, positrons, and photons. For example, we can consider the case where two electrons are propelled towards each other. Using two of the vertex pictured in \ref{Fig:Intro:Vertex1}, we can see that this must result in two electrons, as can be seen in Figure \ref{Fig:Intro:Feynman1}, where time propagates to the right on the x-axis, and distance is represented on the y-axis. 
\begin{figure}[h]
    \centering
        \includegraphics[width=0.49\textwidth]{F1/FeynDiag1}
        \caption{A Feynman Diagram showing the interaction of two electrons. Time increases to the right on the x-axis, and the y-axis represents the distance between the two electrons.}
        \label{Fig:Intro:Feynman1}
\end{figure}
Alternatively, if we were to consider the interaction between an electron and a positron, which can be represented as an electron with an arrow pointing backwards in time, we can see that the resulting process is described by Figure \ref{Fig:Intro:Feynman2}, where an electron and a positron annihilate to a photon, which produces an electron positron pair.
\begin{figure}[h]
    \centering
        \includegraphics[width=0.49\textwidth]{F1/FeynDiag2}
        \caption{A Feynman Diagram showing the annihilation of a positron and an electron.}
        \label{Fig:Intro:Feynman2}
\end{figure}
In both of these diagrams, we notice that charge is conserved on either side of the diagram. Many more diagrams can be made with this fundamental vertex as a building block, combining it with other vertices that involve a photon or electrons as well. In understanding diagrams like these, we can explain three of the four fundamental forces at a qualitative level, below, while also introducing new theories in later chapters that have motivated the search for new particles.

\subsection{The Electromagnetic Force:}
Quantum electrodynamics (QED) describes how the electromagnetic force behaves at the quantum level. The force carrier of QED is the photon, $\gamma$, which is massless and moves at the speed of light. This theory can be described by diagrams like Figure \ref{Fig:Intro:Vertex1}, where any fermion with an electric charge can replace the electron. It should be noted that since the photon has neutral electric charge, it cannot couple to itself, and these diagrams with two electrically charged fermions and one photon represent the theory in full. Many experiments have performed precision measurements of QED properties, verifying that they match SM predictions extremely well.

\subsection{The Strong Force:}
Quantum Chromodynamics (QCD) describes quantum interactions related to the strong force. The force carrier of QCD is the gluon, g, which is massless and moves at the speed of light. The gluon does have color charge, just as the quarks do, and therefore there are a number of vertices in QCD, as shown in Figure \ref{Fig:Intro:Vertex2}.
\begin{figure}[h]
    \centering
        \includegraphics[width=\textwidth]{F1/Vertex2}
        \caption{QCD interactions can be described by three vertices. The curly line represents gluons, while the straight lines represent quarks. Since gluons are colored, they can interact with themselves.}
        \label{Fig:Intro:Vertex2}
\end{figure}
As mentioned before, each quark has a color charge, and each anti-quark has an anti-color charge. Since color must be conserved, we can see that the gluon must carry a color and an anti-color for the vertices in Figure \ref{Fig:Intro:Vertex2} to be true. Naively, since there are three colors, one would think there are nine gluons, or nine possible combination of three colors and three anticolors. These are summarized below:

{\centering

$\frac{r\overline{r} + b\overline{b} + g\overline{g}}{\sqrt{3}}$

}

{\centering

    $\frac{r\overline{b} + b\overline{r}}{\sqrt{2}}$\hspace{1cm}$\frac{r\overline{g} + g\overline{r}}{\sqrt{2}}$\hspace{1cm}$\frac{b\overline{g} + g\overline{b}}{\sqrt{2}}$\hspace{1cm} $\frac{r\overline{r} - b\overline{b}}{\sqrt{2}}$
    
    }

{\centering  

    $\frac{-i(r\overline{b} - b\overline{r})}{\sqrt{2}}$\hspace{1cm}$\frac{-i(r\overline{g} - g\overline{r})}{\sqrt{2}}$\hspace{1cm}$\frac{-i(b\overline{g} - g\overline{b})}{\sqrt{2}}$ \hspace{1cm}$\frac{r\overline{r} + b\overline{b} -2g\overline{g}}{\sqrt{6}}$
    
    }

Both mesons (two-quark bound states) and baryons (three-quark bound states) must be color neutral. We know that a particle can be neutral if it contains either quarks with color-anti-color, or all three color quarks. Then since the combination of red, green, and blue is neutral, the first combination listed, $\frac{r\overline{r} + b\overline{b} + g\overline{g}}{\sqrt{3}}$, would be color neutral. However, color neutral particles must be non-interacting, otherwise colorless baryons would emit these gluons and interact with one another through the strong force, which we do not observe in nature. Therefore, this first combination is not possible, and there are only eight gluons, called the color octet.

The strong force is responsible for hadronization, or the production of many quarks and gluons when quarks are smashed apart due to an event such as a proton-proton collision. As quarks drift apart after the collision, a color tube of self-interacting gluons is created between the quarks. These tubes are stretched as the quarks drift further apart, increasing energy in the tube due to the constant force exerted from stretching. At distances of roughly $10^{-15}$ m, it becomes more energetically favorable for two new quarks to be created from the vacuum, through the process shown in the top left vertex in Figure \ref{Fig:Intro:Vertex2}. If these new quarks are still too energetic to be contained in a particle, this process will repeat until the energy has been sufficiently decreased for all new quark pairs to stay bound. Quark confinement is critical to the understanding of hadronic activity within a particle detector. This means that instead of seeing one hadronic particle in a detector, we see a cone of hadronic activity, called a jet.

\subsection{The Weak Force:}
When the universe had first begun and was still very hot, the weak force and the electromagnetic force were combined to form the electroweak force. At this time, the electroweak interaction was mediated by four massless bosons: $W_1$, $W_2$, $W_3$ and $B$. As the universe cooled, the bosons eventually began interacting with the Higgs field (addressed in the following section), and soon, the bosons that interacted with fermions became a superposition of these four. 
We can write these superpositions as
\begin{equation}
    \left( \begin{array}{c} \gamma \\ Z \end{array} \right) = \left( \begin{array}{c} \cos\theta_W \\ -\sin\theta_W \end{array} \begin{array}{c} \sin\theta_W \\ \cos\theta_W \end{array} \right) \left( \begin{array}{c} B \\ W_3 \end{array} \right)
\label{eqn:WZ1}
\end{equation}
    \begin{equation}
    W^\pm = \frac{(W_1\mp iW_2)}{\sqrt{2}}
\label{eqn:WZ2}
\end{equation}
where $\theta_W$, the mixing angle, is a parameter of the SM. Therefore, while $\gamma$ is the force carrier for the electromagnetic force, the $Z$, $W^{+}$, and $W^{-}$ are the force carriers for the weak force. The weak interaction is interesting because it only acts on left-handed particles and right-handed anti-particles. As mentioned when describing leptons and quarks, nature distinguishes between the handedness of particles, and in particular, weak hypercharge is different for left-handed particles and right-handed particles.

The vertices describing the weak force interactions can be found in Figure \ref{Fig:Intro:Vertex3}.
\begin{figure}[h]
    \centering
        \includegraphics[width=\textwidth]{F1/Vertex3}
        \caption{Weak interactions can be described by the following vertices. The wavy lines represent bosons, while the straight lines represent any weak-interacting particle where the appropriate quantities are conserved.}
        \label{Fig:Intro:Vertex3}
\end{figure}
While the photon is massless, the three weak force carriers are not. The Z boson has a mass of $91.2$ GeV/c$^2$ (for reference, protons have a mass of $\sim$ 1 GeV/c$^2$). The Z boson allows particles to interact with their anti-particles through the weak force. For example, the process $Z\rightarrow e^-e^+$ is permitted, but the process $Z\rightarrow\mu^+e^-$ is not. The Z boson can couple to both electrically-charged and -neutral particles, as well as particles with and without color.

The $W^\pm$ boson, which has a mass of $80.4$ GeV/c$^2$, also couples pairs of fermions. Rather than linking particles and anti-particles like the Z boson, it allows the flavors within a generation to interact. For example, a $W^{+}$ boson can decay to two quarks ($W^{+}\rightarrow u\overline{d}$), and a $W^{-}$ can decay to a lepton and its neutrino ($W^{-}\rightarrow\mu\nu_\mu$). Since the $W^{\pm}$ is electrically charged, electric charge must be conserved at each vertex, and also $\gamma\rightarrow W^+W^-$ is allowed. The decay $Z\rightarrow W^+W^-$ is also a possible vertex.

%The W boson is a fascinating object to study. As we've already mentioned in passing: The weak interaction only acts on left-handed particles and right-handed anti-particles\footnote{This causes considerable complications because neutrinos have mass. It is therefore possible to define a reference frame in which the neutrino is moving backwards with relation to the W it decayed from, thus flipping its helicity...}.

%The decay $\gamma\rightarrow \mu^+e^-$ is a perfectly plausible decay, seeing as charge is conserved. However, it never occurs. Similarly, the Z boson might be allowed to decay to different generation quarks (this wouldn't violate charge or color), but it simply does not happen. These decays are called \textit{flavor changing neutral currents}. Flavor changing \textit{charged} currents on the other hand, do exist. Specifically, Ws can decay to quark pairs outside of their generation. For example, we might see $W\rightarrow s\overline{u}$. 
The probability of a W decaying to different generations is dictated by the Cabibbo-Kobayashi-Maskawa (CKM) matrix, where each of these values has been experimentally measured:
\begin{equation}\label{Eq:CKM}
    \left(\begin{array}{c} V_{ud}\\V_{cd}\\V_{td} \end{array}\begin{array}{c}  V_{us}\\V_{cs}\\V_{ts} \end{array}\begin{array}{c} V_{ub}\\V_{cb}\\V_{tb}  \end{array}\right) = \left(\begin{array}{c} 0.974\\0.225\\0.009 \end{array}\begin{array}{c}  0.225\\0.973\\0.040 \end{array}\begin{array}{c} 0.004\\0.041\\0.999  \end{array}\right)
\end{equation}
where $|V_{ij}|^2$ is the probability that a quark $i$ decays to a quark $j$ through the emission of a W. This matrix demonstrates that intra-generational coupling (for example, u and d) dominates. This also explains why b-quarks have such a long lifetime, since $V_{ub}$ and $V_{cb}$ are both very small. This critical feature of weak interactions allows for easy distinction between hadronic activity containing a b-quark and hadronic activity that has no b-quarks, which we will explain further in Chapter~\ref{Sec:BSM}.

Lastly, since all other particles decay through a W to a lighter particle, this explains why our world is built with the lightest two quarks (up and down quarks) and electrons, the building blocks of atoms. 

\subsection{Mass and the Higgs Boson}
The Higgs Boson is the most recent confirmed Standard Model particle. It is related to the mass of other SM particles. To understand how this is so, we will look at the Higgs potential, which we can write as $V(\phi) \sim (\phi^2 - \eta^2)^2$ for the purpose of explanation. This is plotted in Figure \ref{Fig:Intro:HiggsPot}.
\begin{figure}[h]
    \centering
        \includegraphics[width=0.66\textwidth]{F1/higgspot.png}
        \caption{Approximation of the Higgs potential. While the shape is symmetric about the y-axis, the function is not symmetric about the minima.}
        \label{Fig:Intro:HiggsPot}
\end{figure}
The potential is symmetric about the y-axis, or the axis with units of energy. In the early universe while everything was still very hot, the energies of particles were so high that the bumps at the bottom of the potential had no consequence. As the universe cooled, however, these minima became important, and a minima was chosen. This is known as spontaneous symmetry breaking.

Once the universe had cooled sufficiently, the Higgs field took on a vacuum expectation value (VEV), or an average value in empty space, which was non-zero. This VEV couples to electroweak interactions, as referenced in the previous section, and the photon and weak force bosons mixed to form the states we know today, described in Equations~\ref{eqn:WZ1} and~\ref{eqn:WZ2}. The spontaneously broken Higgs field also couples to itself, quarks, electrons, muons, taus, Z bosons, and $W^{\pm}$ bosons. The magnitude of the coupling between the Higgs and a given particle determines the mass of the particle; the larger the coupling, the more massive the particle. However, it does not couple to neutrinos, so the explanation for their mass must lie elsewhere.

While we have been referring to the Higgs field, the Higgs boson is the particle that was recently discovered at the LHC in 2012. This particle is an excitation of the Higgs field in the same way that a photon is an excitation of the electromagnetic field. The SM Higgs has no charge, no spin, and a mass of 125 GeV. It couples to anything with mass (aside from the neutrinos), so it can also couple to itself. These vertexes take the form shown in Figure \ref{Fig:Intro:Vertex4}.
\begin{figure}[h]
    \centering
        \includegraphics[width=0.5\textwidth]{F1/Vertex4}
        \caption{The Higgs Vertex. The dashed line is the Higgs Boson and the straight lines are any massive particle.}
        \label{Fig:Intro:Vertex4}
\end{figure}

\section{Rounding Up the Standard Model Particles}
In total, including particles of different colors and all anti-particles, there exist 61 particles in the SM. We can describe their interactions easily using the vertices presented in each section. For example, consider the diagrams in Figure \ref{Fig:Intro:hh}.
\begin{figure}[h]
    \centering
        \includegraphics[width=0.5\textwidth]{F1/dihiggs}
        \caption{Left: Two gluons interact with a quark, while each gluon interacts with an additional different quark, and these additional different quarks interact with a shared fourth quark and each pair of quarks interacts to form a Higgs boson. Right: Two gluons interact with a quark, while each gluon interacts with an additional quark, and these additional different quarks interact with each other to form a Higgs boson, which interacts with two Higgs bosons. The result of both of these processes is a final state with two Higgs. For both diagrams, note that the only quark interaction of significance is with t, so q represents t for the most part.}
        \label{Fig:Intro:hh}
\end{figure}
These are important physics processes by which two gluons interact with quarks to produce two Higgs, which is central to the topic of this thesis. This shows an example of the different kind of processes possible using Feynman diagrams: two protons collide and gluons inside of them interact with different quarks to produce two Higgs bosons (left) or one Higgs boson that then produces two Higgs bosons (right). We note that the real particles in these interactions, or the particles that can be observed, are the two gluons that begin the interaction and the two Higgs that are the end result of the interaction. The particles in between are virtual particles, which allow the interaction between the two gluons to take place and produce two Higgs, but cannot be observed.

%Indeed, we can make these diagrams as complicated as we want. Some of them become quite silly. See Figure \ref{Fig:Intro:Diags} for some more entertaining examples.\begin{figure}[h]
%    \centering
%        \includegraphics[width=0.4\textwidth]{F1/goof3}
%        \includegraphics[width=0.4\textwidth]{F1/goof2}
%        \includegraphics[width=0.66\textwidth]{F1/goof1}
%        \caption{Some more Feynman Diagrams representing real physics processes. Some of these play real roles in modern analyses and some of these were constructed for the sake of the diagram itself.}
%        \label{Fig:Intro:Diags}
%\end{figure}

While this description has been qualitative, and has not drawn distinction between what has been experimentally determined and what has been predicted in the SM, the SM is quite self-consistent. There are only 19 parameters which are experimentally determined, most of which are particle masses.


%We've given a mostly qualitative description of the model, without differentiating between facts gleaned from experiments and facts that come from calculations within the model itself. Fortunately, the Standard Model is extremely self-consistent and despite there being 61 particles, 4 forces and 1 Higgs field, there are only 19 ``free parameters'' which have to be determined experimentally. Most of them are masses. The Higgs Boson is the reason that the particles have mass, but it doesn't specify what those masses should be. What was the question ``why does a particle have this mass'' now becomes ``why is this particle's coupling to the Higgs what it is''. The CKM Matrix's nine entries can be interrelated with just four parameters. The remaining parameters have to do with the Higgs field or the couplings of the forces.

%\subsection{The top quark}
%No discussion of the standard model is complete unless it mentions how strange the top quark is\footnote{even though it has strangeness 0.}. The top quark's mass is 174 GeV/c$^2$. The next heaviest quark barely makes it to the 5 GeV/c$^2$ mark, with all other fermions being lighter than it. 

%This might just be a curiosity, as there is nothing inherently ``wrong'' with a very heavy top, but it does lead to some strange behavior. The top quark is the only quark that can exist outside of a color singlet. This is because the color tubes we mentioned before never form: due to its very large mass, the top decays immediately (in $5\times 10^{-25}$ seconds) through the Weak force to a bottom quark (or very rarely, a c or a d) and a W Boson. 

%This does not mean that the top quark can only interact weakly, just that it decays that way. The top quark plays an important role in Higgs production (since the Higgs couples so strongly to it), as shown in Figure \ref{Fig:Intro:TopDecayHiggs}. As we will see, the top quark is a frequent player in searches for new physics as new theories often couple preferentially to massive objects.
%\begin{figure}[h]
 %   \centering
 %       \includegraphics[width=0.66\textwidth]{F1/higgsProd}
 %       \caption{Production of a Higgs boson from the collision of two gluons (which, notably, cannot couple directly to the Higgs). The triangular structure is a \textit{loop}.}
 %       \label{Fig:Intro:TopDecayHiggs}
%\end{figure}
\section{Beyond the Standard Model}
While the SM is an impressive feat of physics, mathematics, and experiments, it leaves many questions unanswered. For example, the SM only accounts for baryonic matter, which is estimated to be $4.6\%$ of the universe. A remaining $24\%$ is accounted for by dark matter, which can only be gravitationally detected, while the rest of the universe is made up of dark energy.

The SM is also missing one of the four fundamental forces, gravity. While we have convincing understanding of quantum mechanics and general relativity, a unification for these two theories has yet to be discovered. Neutrino mass is unaccounted for by the SM. The Higgs mass is much lower than one might expect from the SM, which is related to why gravity is so weak compared to the other SM forces. The universe is dominated by matter, rather than anti-matter. 

Many questions remain that require answers beyond the SM, and so we search for new particles and deviations in SM parameters at the LHC in hopes of providing new insight into a theory that would better explain the universe we observe.