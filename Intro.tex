\chapter{Introduction}\label{Sec:Intro}
\begin{figure}[h]
    \centering
        \includegraphics[width=0.33\textwidth]{F1/Cloud}
        \caption{Photograph of a cloud chamber from 1933: the vertical line is the path of an positron curving in a magnetic field.}
        \label{Fig:Intro:Elec}
\end{figure}

The first hints that there were particles smaller than the atom came in 1869 when Johann Wilhelm Hittorf discovered Cathode Rays\cite{cathode}. It would take a further 42 years before these were indubitably identified as electrons (and photographed, see Figure \ref{Fig:Intro:Elec}). This was the world's first subatomic particle. The proton came soon after (1917) and the neutron a little later (1935).

However, as quantum mechanics developed, a number of new particles came onto the scene. Some were ``needed'' for models to make sense\footnote{the pion for example} and others appeared un-invited\footnote{such as the muon, whose discovery in 1937 prompted the famous quip ``who ordered that?''}.  Before long, there were hundreds of subatomic particles and an entire branch of physics dedicated to their study. The Particle Data Group now publishes a text-book sized reference\cite{PDG-2014} listing the masses, lifetimes, charges, spins, strangenesses, isospins, charms, and couplings of the $\psi$, $\Delta^-$, $\pi^+$, $\mu$, $\Omega_{ccc}^{++}$ and hundreds of other particles. Fortunately, all of these fit within a mathematical framework called \textit{The Standard Model}.

\section{The Standard Model}
The Standard Model is sometimes called a ``Theory of Almost Everything''\footnote{This is extremely inaccurate...}. It describes the hundreds of particles discovered and explains everything about them and how they interact with a single formula (shown in Figure \ref{Fig:Intro:SML}). Amazingly, this formula can be boiled down to a simple list of particles and mediating forces usually represented by the diagram in Figure \ref{Fig:Intro:SMTable}.
\begin{figure}[h]
    \centering
        \includegraphics[width=0.75\textwidth]{F1/StandardModelEquation}
        \caption{The Lagrangian of the Standard Model.}
        \label{Fig:Intro:SML}
\end{figure}
\begin{figure}[h]
    \centering
        \includegraphics[width=\textwidth]{F1/StandardModelTable}
        \caption{The Standard Model. Image taken from CERN\cite{SMTable}.}
        \label{Fig:Intro:SMTable}
\end{figure}

\clearpage
\subsection{The Leptons}
The electron retains its status as a truly fundamental particle. It is joined by two heavy versions of itself: the muon ($\mu$) and the tau ($\tau$), roughly 200 and 4,000 times as massive as the electron. Each is negatively charged\footnote{We say an electron has charge -1: which corresponds to - 1.602 $\times$ 10$^{-19}$ Coulombs}. 

These particles are each paired with a neutrino of a given flavor, for example, the electron neutrino is partnered with the electron; they are denoted $\nu_e$, $\nu_\mu$ and $\nu_\tau$. The neutrinos are chargeless and massless\footnote{This is not true: While the Standard Model neutrino has no mass, various experiments have shown that this is not the case. An explanation for neutrino mass must lie outside of the Standard Model.}. 

Each of these particles has an anti-particle, with opposite charge (denoted by a bar, or by specifying the charge: $\overline{e}$, $\overline{\mu}$, $\overline{\tau}$ or $e^+$, $\mu^+$, $\tau^+$). Neutrinos, while they are chargeless, also have antiparticles: the anti-neutrinos ($\overline{\nu}_e$, $\overline{\nu}_\mu$ and $\overline{\nu}_\tau$).

There is another property we must mention: the \textit{spin}. Spin is the intrinsic angular momentum of the particle. The electron is a point particle, and there is therefore nothing actually revolving. Just as in the macro-world, there are two directions something can spin around an axis, which we denote as positive or negative spins. The spin is a vector quantity. The leptons all have spin $\frac{1}{2}$ (or $-\frac{1}{2}$), which means they are \textit{fermions}. 

The spin of a particle allows us to define one more property: the \textit{helicity} (also called \textit{handedness}). The helicity is defined as \textit{the sign of the projection of the particle's spin vector onto its momentum vector}. We usually refer to negative helicity as left-handedness and positive helicity as right-handedness. Handedness might seem like a trivial quantity but it plays an important part in the actual calculations that are part of the Standard Model: left and right handed particles can behave differently. Indeed, while the massive leptons can be either left or right handed, it appears that all neutrinos are left-handed, and all anti-neutrinos are right handed. There is currently no explanation for this strange fact.

\subsection{The Hadrons}
The Standard Model constructs the myriad particles found in the last century (excluding of course the leptons) out of just six particles, called the quarks. Like the leptons they exist in three separate groups (called ``generations''). They are: the \textit{up} ($u$) and \textit{down} ($d$) quarks, the \textit{strange} ($s$) and \textit{charm} ($c$) quarks, and the \textit{bottom} ($b$) and \textit{top} ($t$) quarks.

The $u$, $c$ and $t$ quarks have charge $+\frac{2}{3}$ and the $d$, $s$ and $b$ quarks have charge $-\frac{1}{3}$. Each has its own mass, and like the leptons they have antiparticles ($\overline{u}$,$\overline{d}$,$\overline{s}$,$\overline{c}$,$\overline{b}$ and $\overline{t}$), and spin. Like the leptons they have spin  $\frac{1}{2}$ (or $-\frac{1}{2}$) and are therefore also fermions.

Unlike the leptons however, the quarks are colored\footnote{this is just a different kind of charge, it has nothing to do with the colors we experience visually}. While there are two ``kinds'' of electric charge: positive and negative, color is a bit more complicated: quarks can be \textit{red}, \textit{green} or \textit{blue}. Anti-quarks meanwhile can be \textit{anti-red}, \textit{anti-green} or \textit{anti-blue}.

The electric charge can be detected directly, color cannot. Because of the particularities of the force governing colored particles, the quarks always come bound together in color neutral combinations, either by combining all three colors (which count as neutral) or by combining a color and its anticolor. For example, the proton is the combination of two $d$ and one $u$ quarks. Each of these caries one of the distinct colors and the combination is neutral. The pion ($\pi^+$) is a combination of a $u$ and a $\overline{d}$ where the up quark would carry some color and the (anti-)down would carry that same anti-color. All the particles discovered in the 1900s were just combinations of the six quarks. The fractional charges of the quarks can never be directly detected as there are no colour neutral combinations which do not result in unit charge.

\subsection{The Forces}
Without some forces to hold all these pieces together, the universe would be a soup of quarks and leptons. In the Standard Model, forces are not ``Actions at a Distance'', but rather, the exchange of \textit{force mediating} particles.

Consider the most basic of particle interactions: two electrons are shot towards each other. Since they are both negatively charged, they repel each other. If we consider that the system starts in some state $i$ and ends in some state $f$, the probability of the transition $i\rightarrow j$ happening is related to the infinite sum:
\begin{equation}
    S_{fi} = \sum_{n=0}^{\infty}S^n
\end{equation}
The sum is over orders of perturbation\footnote{Quantum Mechanics uses perturbation theory to move from a simple approximate solution to a more precise one.}, where the first order $n=1$ state is the case where the electrons do not interact at all. The next (and first interesting) term of S is:
\begin{equation}
    S^{\{2\}} = \pm^{\{n\}}\bar\psi(x)ie\gamma^\mu\psi(x)\bar\psi(x^\prime)ie\gamma^\nu\psi(x^\prime)\int\frac{d^4k}{(2\pi)^4}\frac{-g_{\mu\nu}}{k^2-i0}e^{-ik(x-x^\prime)}
\end{equation}
where the integral is over all the possible momenta of the incoming and outgoing electrons. This is just the first non-trivial term of the sum. Fortunately it is the dominant term. Unfortunately, a clear description of the process is hidden by the opaqueness of the mathematics. Enter the \textit{Feynman diagram}. Each term in the expression for $S^{\{2\}}$ can be encoded in as a pictoral element of a Feynman Diagram (as shown in Figure \ref{Fig:Intro:Feynman0}). 
\begin{figure}[h]
    \centering
        \includegraphics[width=0.49\textwidth]{F1/Fee}
        \caption{$S^{\{2\}}$ represented as a Feynman Diagram. Each term in the equation corresponds to either a propagator (a line) or a verted (a ``corner''). The terms \underline{$A_\mu(x)A_\nu(x^\prime)$} is the integral term.}
        \label{Fig:Intro:Feynman0}
\end{figure}
With the Feynman Diagram, we can explicitly refer to physical processes with an (exact) visual description. There is no loss of generality or rigor!

Returning to the specifics of our example; two electrons interacting: the entire (first order) interaction is encoded in the Feynman Diagram in Figure \ref{Fig:Intro:Feynman1}. The effect of the electromagnetic force is just the exchange of a photon\footnote{This model is of course, only valid on the very small scales (both in time and space) of particle physics. If we were to really shoot two electrons together in a laboratory, they would exchange a great many photons, not just one.}. 
\begin{figure}[h]
    \centering
        \includegraphics[width=0.49\textwidth]{F1/FeynDiag1}
        \caption{A Feynman Diagram depicting the interaction of two electrons. The action of the electromagnetic force is equivalent to the exchange of a photon. The vertical axis represents the distance between the electrons and the horizontal axis the passage of time.}
        \label{Fig:Intro:Feynman1}
\end{figure}
The Standard Model however doesn't just describe how particles interact. Indeed, the interaction we've just described could be described without any particle physics or even quantum mechanics. What the Standard Model allows us to do that previous formulations did not was account for the fact that particles can be created or destroyed. Consider for example the diagram in Figure \ref{Fig:Intro:Feynman2}. 
\begin{figure}[h]
    \centering
        \includegraphics[width=0.49\textwidth]{F1/FeynDiag2}
        \caption{A Feynman Diagram depicting the annihilation of a positron and an electron.}
        \label{Fig:Intro:Feynman2}
\end{figure}
Here, an electron ($e^-$) and an anti-electron ($e^+$) are shot at each other. They annihilate, which is to say: they become (or merge into) a photon. Some time later, the photon decays into a new set of positrons and electrons. During this interaction charge was conserved, since the photon is chargeless and the electron/positron pair combined has $1 + -1 = 0$ charge.

These two processes are essentially equivalent. They are just different arrangements of the same \textit{vertex}. A vertex is a point interaction between particles. In the case of electron-positron annihilation or electron-electron repulsion, the vertex is the same, and is shown in Figure \ref{Fig:Intro:Vertex1}.
\begin{figure}[h]
    \centering
        \includegraphics[width=0.33\textwidth]{F1/Vertex1}
        \caption{Simple electromagnetic vertex describing the interaction of two electrons and a photon.}
        \label{Fig:Intro:Vertex1}
\end{figure}
With just this vertex, we can construct a huge number of possible electromagnetic interactions. In the diagrams, electrons and positrons are the same: a positron moving backward in time (in the negative time direction on the axis, so to the left in our diagrams) is the same as an electron moving forward in time. In the case of the photon, similar vertexes exist for any charged particle and it's (opposite sign) anti-particle. For example, muons can be produced via the process $e^+e^-\rightarrow\gamma\rightarrow\mu^+\mu^-$, whose Feynman diagram is shown in Figure \ref{Fig:Intro:Feynman3}.
\begin{figure}[h]
    \centering
        \includegraphics[width=0.33\textwidth]{F1/Feynman3}
        \caption{Annihilation of an electron and a positron to create a muon and anti-muon. Backwards going arrows are often used in lieu of charges to denote anti-particles.}
        \label{Fig:Intro:Feynman3}
\end{figure}
Now that we are armed with the Feynman Diagram, each force can be thought of as some mediator particles with some rules governing which vertexes may be constructed. We will allow the Feynman Diagrams to obfuscate the computations, and focus on a qualitative description of these processes.

\subsection{The Electromagnetic Force:}
The field theory description of the Electromagnetic Force is called \textit{quantum electrodynamics}, often abbreviated as \textit{QED}. It describes how charges interact. QED with its carrier particle, the photon ($\gamma$), is the simplest of the fundamental forces described by the Standard Model. In fact, we've already described it fully, as the only vertex included is that in Figure \ref{Fig:Intro:Vertex1}. The photon is massless, and moves at the speed of light. 

QED has been rigorously tested by a variety of experiments, with very little deviation from the behavior expected from the Standard Model\footnote{very little deviation here means truely miniscule deviations: differences on the order of 10/1,000,000,000ths of the predicted values, and well within the expected uncertainties of such measurements.}. 

\subsection{The Strong Force:}
\textit{Quantum Chromodynamics} (abbreviated to \textit{QCD}) is the field theory of colored interactions. The force carrier is the \textit{gluon} (g). The photon is chargeless, so it cannot couple to itself. The gluon however carries color; which means that there is more than one vertex in QCD. These vertexes are shown in Figure \ref{Fig:Intro:Vertex2}.
\begin{figure}[h]
    \centering
        \includegraphics[width=\textwidth]{F1/Vertex2}
        \caption{Three vertexes arising from QCD interactions. The curly line represents gluons. Note that gluons, which are colored, may interact with themselves.}
        \label{Fig:Intro:Vertex2}
\end{figure}

Each quark carries one color (and each anti-quark carries one anti-color). Color is conserved, so for the diagrams in Figure \ref{Fig:Intro:Vertex2} to work, the gluons must carry both a color and an anti-color. There is no $r\overline{r}$ (red + anti-red) gluon. Instead, we must delve a little bit into the behavior of the strong force.

While the protons and the neutrons (called neucleons) are confined to nuclei by the strong force, they are not colored the way the quarks are. This might seem like a contradiction, but it is not: the gluon is not the carrier particle which binds the neucleons together. Instead, the neucleons exchange mesons (two-quark bound states) in the same way that two electrons might exchange photons. This is sometimes called the \textit{residual strong force}, it's an emergent property of complex systems of colored particles. While the gluon is massless (and therefore, like the photon, not limited in range) the mesons are not, limiting the range over which this interaction can take place, and explaining why the strong force can't be felt the same way as the electric force.

The leptons are truly ``colorless'', but while the protons and the neutrons behave similarly, they are a \textit{color singlet}. The color singlet is the superposition\footnote{In quantum mechanics, the results of measurements can be a superposition of states: for example, the color state $(r\overline{b} + b\overline{r})/\sqrt{2}$ corresponds to a gluon which is equally likely to be in the state red + anti-blue or blue + anti-red. The square root of two is a normalizing term to make the quantum mechanical calculations return probabilities less than one.} of the quantum states:
\begin{equation}
    \frac{r\overline{r} + b\overline{b} + g\overline{g}}{\sqrt{3}}
\end{equation}
Singlets may interact with each other (this is what's happening when a pion travels between two neucleons). Since singlets may interact with each other, and there is no mass to limit the range of the gluon, we can be sure (empirically) that no gluons carry the singlet color state.

Gluons can take any of the remaining color combinations. Instead of writing them all down, we represent them with a linearly independent set from which all of them can be constructed: 

{\centering

    $\frac{r\overline{b} + b\overline{r}}{\sqrt{2}}$\hspace{1cm}$\frac{r\overline{g} + g\overline{r}}{\sqrt{2}}$\hspace{1cm}$\frac{b\overline{g} + g\overline{b}}{\sqrt{2}}$\hspace{1cm} $\frac{r\overline{r} - b\overline{b}}{\sqrt{2}}$
    
    }

{\centering  

    $\frac{-i(r\overline{b} - b\overline{r})}{\sqrt{2}}$\hspace{1cm}$\frac{-i(r\overline{g} - g\overline{r})}{\sqrt{2}}$\hspace{1cm}$\frac{-i(b\overline{g} - g\overline{b})}{\sqrt{2}}$ \hspace{1cm}$\frac{r\overline{r} + b\overline{b} -2g\overline{g}}{\sqrt{6}}$
    
    }
    
These are the color \textit{octet}. It is not the only such set possible, but there are no combinations which are less complex. The octet must be constructed in such a way that it is impossible to get the singlet state.

What does happen if we try to forcibly separate a singlet state? For example, let's say that we smash an electron into a proton with sufficient force to temporarily overcome the strong force and one of its quarks is ejected?

The shards of the proton coming out of this collision all carry color, and are therefore able to interact with each other. If they only felt electric charges, they would simply exchange photons. Not so with the strong force: as the quarks drift away form each other, the carrier gluons, which can interact with each other) form a web between the them. We call this object a \textit{color tube}. The tube containing a jumble of self-interacting gluons. These tubes exert approximately constant force when stretched, increasing the energy in the tube until\footnote{at distances of about $10^{-15}$, incidentally the roughly the radius of an atomic nucleus} it becomes more energetically favorable to create a new pair of quarks out of the vacuum (through the top left vertex shown in Figure \ref{Fig:Intro:Vertex2}).

The result is that the quark that was drifting away, abhorrently not contained in a singlet state, suddenly finds itself paired with a new quark. If there isn't too much energy, these now stay bound and have become a meson. If the quark is still too energetic to be contained, then the process will repeat itself, chipping away at the energy of the original quark until every colored particle is collected in a singlet state. This property is called \textit{containment}, and the process by which new quarks are produced to enforce it is called \textit{hadronization}. 

Hadronization plays a very important part in the experiment this Thesis will describe. It's relationship with particle detection is described in detail in Chapter \ref{Sec:CMS}.

\subsection{The Weak Force:}
In this very distant past, the Weak force and the Electromagnetic force were one. They were called the Electroweak Interaction, and there were four carrier particles, $W_1$, $W_2$, $W_3$ and $B$. All four of these were massless, and had unit spins (as such, they were \textit{bosons}). While temperatures were high enough, these bosons were able to ignore the effects of the Higgs field, which we discuss in the next section.

As the universe cools down, the four bosons begin to interact with this field. The fermions now no longer interact directly with the $W_1$, $W_2$, $W_3$ and $B$, but rather, with superpositions of them. Where previously there were a $B$ and a $W_3$ Boson, we now have, $\gamma$ and $Z$. This superposition is simple to write:
\begin{equation}
    \left( \begin{array}{c} \gamma \\ Z \end{array} \right) = \left( \begin{array}{c} \cos\theta_W \\ -\sin\theta_W \end{array} \begin{array}{c} \sin\theta_W \\ \cos\theta_W \end{array} \right) \left( \begin{array}{c} B \\ W_3 \end{array} \right)
\end{equation}
Where $\theta_W$ called the Mixing Angle or Weinberg Angle, is a parameter of the Standard Model. The photons of the Electromagnetic force are just a particular combination of the $B$ and $W_3$. The $Z$ Boson is one of the carrier particles of the Weak Force. Similarly ,the $W_1$ and $W_2$ bosons combine via the superposition
\begin{equation}
    W^\pm = \frac{(W_1\mp iW_2)}{\sqrt{2}}
\end{equation}
to form two new Bosons, the $W^+$ and $W^-$. 

We've already talked about the behavior of the photon, so we just need to detail the vertexes involving the W and Z bosons for a full description of the Weak Force. These vertexes are shown in Figure \ref{Fig:Intro:Vertex3}.
\begin{figure}[h]
    \centering
        \includegraphics[width=\textwidth]{F1/Vertex3}
        \caption{Vertexes arising from weak interactions. The wavy lines represent the Bosons. The straight lines represent any particle where the appropriate quantities are conserved (details in the text).}
        \label{Fig:Intro:Vertex3}
\end{figure}
The Z Boson is not massless, it has mass $91.2$ GeV/c$^2$\footnote{The unit GeV/c$^2$ is a unit of mass. One eV/c$^2$ is equal to $1.78 \times 10^{-36}$ kg. An electron has mass $\sim$ 0.5 MeV/c$^2$ and a proton $\sim$ 1 GeV/c$^2$.}. Z bosons couples particles to their anti-particles. For example, the process $Z\rightarrow e^-e^+$ is allowed but the process $Z\rightarrow\mu^+e^-$ is not. Similarly, $Z\rightarrow b\overline{b}$ is allowed but $Z\rightarrow b\overline{u}$ is not. In this regard, it behaves like a heavy version of the photon, except that it can couple to neutral particles (for example, $Z\rightarrow\nu_e\overline{\nu}_e$).

The W boson (with a mass of $80.4$ GeV/c$^2$) also couples pairs of fermions, except that it links the flavors within a generation. For example, a W boson might decay to two quarks ($W\rightarrow u\overline{d}$) or to a lepton and its neutrino ($W\rightarrow\mu\nu_\mu$). It is charged, so the process $\gamma\rightarrow W^+W^-$ is allowed. The decay $Z\rightarrow W^+W^-$ is also allowed.

The W boson is a fascinating object to study. As we've already mentioned in passing: The weak interaction only acts on left-handed particles and right-handed anti-particles\footnote{This causes considerable complications because neutrinos have mass. It is therefore possible to define a reference frame in which the neutrino is moving backwards with relation to the W it decayed from, thus flipping its helicity...}.

The decay $\gamma\rightarrow \mu^+e^-$ is a perfectly plausible decay, seeing as charge is conserved. However, it never occurs. Similarly, the Z boson might be allowed to decay to different generation quarks (this wouldn't violate charge or color), but it simply does not happen. These decays are called \textit{flavor changing neutral currents}. Flavor changing \textit{charged} currents on the other hand, do exist. Specifically, Ws can decay to quark pairs outside of their generation. For example, we might see $W\rightarrow s\overline{u}$. The probability of a W decaying to different generations is encoded in the CKM Matrix, an empirically measured set of values. Here is that matrix:
\begin{equation}\label{Eq:CKM}
    \left(\begin{array}{c} V_{ud}\\V_{cd}\\V_{td} \end{array}\begin{array}{c}  V_{us}\\V_{cs}\\V_{ts} \end{array}\begin{array}{c} V_{ub}\\V_{cb}\\V_{tb}  \end{array}\right) = \left(\begin{array}{c} 0.974\\0.225\\0.009 \end{array}\begin{array}{c}  0.225\\0.973\\0.040 \end{array}\begin{array}{c} 0.004\\0.041\\0.999  \end{array}\right)
\end{equation}
where $|V_{ij}|^2$ is the probability that a quark $i$ decays to a quark $j$ through the emission of a W.

It is then because of the W and the W only that our world seems to be composed of only up and down quarks and electrons. All other particles will eventually decay (through a W) to something lighter. The other leptons (the muons and the taus) will eventually decay by emitting a neutrino and a W, with the W decaying to an electron and a neutrino. All the various particles discovered in the last century eventually decayed to protons and neutrons because the higher order quarks embedded in them decayed (again, through a W) back to the first generation.

\subsection{Mass and the Higgs Boson}
The Higgs Boson is the most recent addition to the Standard Model to be discovered. It is responsible for giving the other particles their mass\footnote{except the neutrinos, which by now you must have noticed, are devious...} As we already mentioned, this does not happen at very high energies. To understand this, we should consider the Higgs Potential. For the purpose of illustration we can simplify it to the form $\sim (\phi^2 - \eta^2)^2$, where $\phi$ is the Higgs Field, which we plot in Figure \ref{Fig:Intro:HiggsPot}\footnote{This shape is sometimes called the ``mexican hat'' or ``champagne bottle'' potential, depending on the level of cultural sensitivity required.}. 
\begin{figure}[h]
    \centering
        \includegraphics[width=0.66\textwidth]{F1/higsspot}
        \caption{Sketch of the Higgs Potential. Notice that while the over-all shape is symmetric about the y-axis, the function is not symmetric about the minima.}
        \label{Fig:Intro:HiggsPot}
\end{figure}
The potential in question is symmetric (about the y-axis, which in the case of the actual Higgs potential would have units of energy). In the high-temperature early universe, the energies of particles was such that the "bump" at the bottom of the potential played no part. As the universe cools, however, things must settle into either of the two valleys. It's not the case that the equations of of the Standard Model aren't symmetric, it's that practically, at the energies in question, they cause behavior which is not. This is called \textit{spontaneous symmetry breaking}.

This asymmetry causes the Higgs field in the low temperature universe to take on a vacuum expectation value (roughly speaking: its average value in empty space) which is non-zero (it's 246 GeV). We usually abbreviate this as just the \textit{VEV}. The VEV couples to the electroweak interactions, creating the photon and the weak force bosons as we experience them.

Through a similar but not identical process, this spontaneously broken Higgs field also couples to the weak bosons, quarks, electron, muon and tau, giving them mass where without it they were massless. The neutrinos do not couple, and are therefore massless in the standard model.

So far we've been talking about the Higgs field, but as you may have heard, there is also a Higgs Boson. This particle is just an excitation of the Higgs Field the way we think of a photon as being an excitation of the electromagnetic field. It was discovered in 2012 at the Large Hadron Collider after a 40 year long search. This Higgs has no charge, has no spin, and has a mass of 126 GeV. It couples to anything with mass, which means it couples to itself. The Higgs Vertex is the very last piece of the Standard Model, and is shown in Figure \ref{Fig:Intro:Vertex4}.
\begin{figure}[h]
    \centering
        \includegraphics[width=0.5\textwidth]{F1/Vertex4}
        \caption{The Higgs Vertex. The dashed line is the Higgs Boson and the straight lines represent any massive particle (including the Higgs).}
        \label{Fig:Intro:Vertex4}
\end{figure}
\section{Rounding Up the Standard Model Particles}
So to recap, there are, once we count all the colors and anti-particles, 61 particles in the standard model. We know they interact, and we can pictorally express a number of complex interactions using the Feynman diagram. For example, consider the diagram in Figure \ref{Fig:Intro:ttbar}.
\begin{figure}[h]
    \centering
        \includegraphics[width=0.5\textwidth]{F1/ttbarIntroDiag}
        \caption{Collider Example: two quarks collide and form a gluon, which decays to two top quarks.}
        \label{Fig:Intro:ttbar}
\end{figure}
This is an important physics process\footnote{incidentally, the primary background of this Thesis' analysis} found in high energy colliders, and we'll discuss it in much more detail later on. For now, it serves as a useful example of the kind of processes we can construct with our Feynman diagrams: two protons collide and quarks inside of them annihilate\footnote{See Chapter \ref{Sec:CMS} for a more formal description of what's happening there} into a gluon. That gluon propagates some distance before it decays back to quarks, this time to a pair of top quarks. Each of these decays through Weak interactions to a b and a W (this is almost always the case, see Equation \ref{Eq:CKM}). One W decays to a lepton and a neutrino, and the other to a pair of quarks. This diagram then contains five distinct vertexes.

Indeed, we can make these diagrams as complicated as we want. Some of them become quite silly. See Figure \ref{Fig:Intro:Diags} for some more entertaining examples.\begin{figure}[h]
    \centering
        \includegraphics[width=0.4\textwidth]{F1/goof3}
        \includegraphics[width=0.4\textwidth]{F1/goof2}
        \includegraphics[width=0.66\textwidth]{F1/goof1}
        \caption{Some more Feynman Diagrams representing real physics processes. Some of these play real roles in modern analyses and some of these were constructed for the sake of the diagram itself.}
        \label{Fig:Intro:Diags}
\end{figure}

We've given a mostly qualitative description of the model, without differentiating between facts gleaned from experiments and facts that come from calculations within the model itself. Fortunately, the Standard Model is extremely self-consistent and despite there being 61 particles, 4 forces and 1 Higgs field, there are only 19 ``free parameters'' which have to be determined experimentally. Most of them are masses. The Higgs Boson is the reason that the particles have mass, but it doesn't specify what those masses should be. What was the question ``why does a particle have this mass'' now becomes ``why is this particle's coupling to the Higgs what it is''. The CKM Matrix's nine entries can be interrelated with just four parameters. The remaining parameters have to do with the Higgs field or the couplings of the forces.

\subsection{The top quark}
No discussion of the standard model is complete unless it mentions how strange the top quark is\footnote{even though it has strangeness 0.}. The top quark's mass is 174 GeV/c$^2$. The next heaviest quark barely makes it to the 5 GeV/c$^2$ mark, with all other fermions being lighter than it. 

This might just be a curiosity, as there is nothing inherently ``wrong'' with a very heavy top, but it does lead to some strange behavior. The top quark is the only quark that can exist outside of a color singlet. This is because the color tubes we mentioned before never form: due to its very large mass, the top decays immediately (in $5\times 10^{-25}$ seconds) through the Weak force to a bottom quark (or very rarely, a c or a d) and a W Boson. 

This does not mean that the top quark can only interact weakly, just that it decays that way. The top quark plays an important role in Higgs production (since the Higgs couples so strongly to it), as shown in Figure \ref{Fig:Intro:TopDecayHiggs}. As we will see, the top quark is a frequent player in searches for new physics as new theories often couple preferentially to massive objects.
\begin{figure}[h]
    \centering
        \includegraphics[width=0.66\textwidth]{F1/higgsProd}
        \caption{Production of a Higgs boson from the collision of two gluons (which, notably, cannot couple directly to the Higgs). The triangular structure is a \textit{loop}.}
        \label{Fig:Intro:TopDecayHiggs}
\end{figure}
\section{A Few Open Questions}
For a supposed Theory of Everything, the Standard Model could use some improvement. It only covers a slim $4.6\%$ of the content of the known universe. Dark Matter, something completely outside the Standard Model, accounts for a further $24\%$, with the rest being mysterious Dark Energy. Both these Dark Entities loom as considerable proof that while the Standard Model does a good job of describing that little $4.6\%$ we call \textit{Baryonic Matter}, it is far from a complete description of the world.

Actually, the Standard Model isn't even a complete description of Baryonic Matter. The obvious problem being of course, that it fails to explain, or in any way even include, the most obvious of forces: Gravity. Gravity is not included in the Standard Model. It's easy to just say that gravity just doesn't matter at the very small distances the Standard Model applies to, but the addition of some massless force-carrying \textit{Graviton} would certainly be welcome.

Of course, there's also these pesky neutrinos and their unwarranted masses to directly (experimentally) challenge our model. These are not the only hints that we're missing something. One obvious observation is that while the Standard Model treats positrons and electrons (and quarks and anti-quarks) symmetrically, the universe's Baryonic Matter is overwhelmingly composed of electrons and quarks. What happened to all the anti-particles?

And what are these insanely energetic cosmic rays we detect from time to time? The Standard Model puts limits on the energy a cosmic ray can attain before it is slowed down by interactions in space... and yet, we have detected particles some orders of magnitude above that limits.

One of the great strengths of the Standard Model is that it is incredibly self-consistent. Indeed, it has been tested and re-tested, with no obvious internal flaw. Now however, as we start to seek the answer to new questions, this strength becomes a flaw as there is no way to modify the model without un-hinging some other internal aspect. To answer the questions of modern physics then, we must venture into new territory Beyond The Standard Model!