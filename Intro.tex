\chapter{Introduction}\label{Sec:Intro}
\begin{figure}[h]
    \centering
        \includegraphics[width=0.33\textwidth]{F1/Cloud}
        \caption{Snapshot of a cloud chamber, circa 1933: the vertical line represents a positron traveling through the cloud chamber, curving due to the magnetic field applied to the chamber.}
        \label{Fig:Intro:Elec}
\end{figure}

In 1869, Johann Wilhelm Hittorf discovered Cathode Rays\cite{cathode}, suggesting that there may be subatomic particles. By 1911, these were identified as electrons, observed in many different ways, including cloud chambers, as can be seen in Figure \ref{Fig:Intro:Elec}. Soon after, protons were discovered in 1917, followed by neutrons in 1935.

As physicists advanced the study of quantum mechanics, many new particles were discovered. These particles can be broken down into several elementary particles, whose interactions are governed by \textit{The Standard Model} (SM).

\section{The Standard Model}
The SM describes the interaction of all elementary particles with one formula, shown in Figure \ref{Fig:Intro:SML}. This lengthy Lagrangian can be summarized as a list of particles and mediating forces, represented in Figure \ref{Fig:Intro:SMTable}.
\begin{figure}[h]
    \centering
        \includegraphics[width=0.75\textwidth]{F1/StandardModelEquation}
        \caption{The Lagrangian of the Standard Model.}
        \label{Fig:Intro:SML}
\end{figure}
\begin{figure}[h]
    \centering
        \includegraphics[width=\textwidth]{F1/StandardModelTable}
        \caption{The Standard Model. Image taken from CERN\cite{SMTable}.}
        \label{Fig:Intro:SMTable}
\end{figure}

\clearpage
\subsection{Leptons}
The electron is one of the most well known elementary particles. There exist two heavier ``versions" of the electron: the muon ($\mu$) which is $\sim 200$ times as massive as the electron, and the tau ($\tau$) which is $\sim 4000$ times as massive as the electron. Each of these particles is negatively charged, where the charge is equivalent to -1 which corresponds to - 1.602 $\times$ 10$^{-19}$ Coulombs.

For each of these particles, there exists a corresponding neutrino, so we have $\nu_e$, $\nu_\mu$ and $\nu_\tau$. The neutrinos as defined by the SM are chargeless and massless, although experiments have proved that at least two out of three neutrinos observed in nature have mass. 

A corresponding anti-particle exists for each of these six particles: $\overline{e}$ ($e^+$), $\overline{\mu}$ ($\mu^+$,), $\overline{\tau}$ ($\tau^+$), $\overline{\nu}_e$, $\overline{\nu}_\mu$, and $\overline{\nu}_\tau$. Note that, while neutrinos are chargeless, the anti-particle for the electron, muon, and tau, all have positive (or opposite of their corresponding particle) charge.

In addition to charge and mass, particles also have spin, or the intrinsic angular momentum of a particle. All of the above mentioned particles, known as leptons, have spin $\frac{1}{2}$ (or $-\frac{1}{2}$), which means they are fermions. The spin of a particle allows us to define helicity, also known as handedness. This is the sign of the projectino of the particle's spin vector onto its momentum vector. A particle is said to be right-handed if the spin and momentum align, and left-handed if the spin and momentum point opposite each other. In the Standard Model, there can be difference between left-handed and right-handed particles. For example, we have observed that all neutrinos are left-handed, while all anti-neutrinos are right-handed, which is currently still inexplicable.

\subsection{Quarks}
Many of the particles discovered in the twentieth century are composite particles called hadrons, made up of quarks. There are six quarks: t\textit{up} ($u$),\textit{down} ($d$), \textit{strange} ($s$), \textit{charm} ($c$), \textit{bottom} ($b$), and \textit{top} ($t$).

The $u$, $c$ and $t$ quarks have charge $+\frac{2}{3}$ and the $d$, $s$ and $b$ quarks have charge $-\frac{1}{3}$. Quarks also have each a different mass. For every quark there is an anti-quark: $\overline{u}$,$\overline{d}$,$\overline{s}$,$\overline{c}$,$\overline{b}$ and $\overline{t}$. Quarks are also fermions, and therefore have a spin of $\frac{1}{2}$ or $-\frac{1}{2}$.

In addition to charge (by which we mean electric charge), mass, and spin, quarks also have a second kind of charge, known as color. While electric charge allows particles to interact through the electromagnetic force, color charge allows particles to interact through a different type of force, called the strong force. Quarks can be red, green, or blue colored, and anti-quarks can be anti-red, anti-green, or anti-blue colored. In nature, we only see color neutral particles, which is why quarks are only observed as combinations forming composite particles. These composite particles made up of quarks are color neutral because they either contain one quark of each color, or a quark of a particular color and a quark of the corresponding anti-color. For example, the proton is made up of on $u$ and two $d$ quarks, so one of these must be red, one must be green, and one must be blue.

\subsection{Representation of Interactions}

%Consider the most basic of particle interactions: two electrons are shot towards each other. Since they are both negatively charged, they repel each other. If we consider that the system starts in some state $i$ and ends in some state $f$, the probability of the transition $i\rightarrow j$ happening is related to the infinite sum:
%\begin{equation}
%    S_{fi} = \sum_{n=0}^{\infty}S^n
%\end{equation}
%The sum is over orders of perturbation\footnote{Quantum Mechanics uses perturbation theory to move from a simple approximate solution to a more precise one.}, where the first order $n=1$ state is the case where the electrons do not interact at all. The next (and first interesting) term of S is:
%\begin{equation}
%    S^{\{2\}} = \pm^{\{n\}}\bar\psi(x)ie\gamma^\mu\psi(x)\bar\psi(x^\prime)ie\gamma^\nu\psi(x^\prime)\int\frac{d^4k}{(2\pi)^4}\frac{-g_{\mu\nu}}{k^2-i0}e^{-ik(x-x^\prime)}
%\end{equation}
%where the integral is over all the possible momenta of the incoming and outgoing electrons. This is just the first non-trivial term of the sum. Fortunately it is the dominant term. Unfortunately, a clear description of the process is hidden by the opaqueness of the mathematics. Enter the \textit{Feynman diagram}. Each term in the expression for $S^{\{2\}}$ can be encoded in as a pictoral element of a Feynman Diagram (as shown in Figure \ref{Fig:Intro:Feynman0}). 
%\begin{figure}[h]
 %   \centering
%        \includegraphics[width=0.49\textwidth]{F1/Fee}
%        \caption{$S^{\{2\}}$ represented as a Feynman Diagram. Each term in the equation corresponds to either a propagator (a line) or a verted (a ``corner''). The terms \underline{$A_\mu(x)A_\nu(x^\prime)$} is the integral term.}
%        \label{Fig:Intro:Feynman0}
%\end{figure}
%Now that we are armed with the Feynman Diagram, each force can be thought of as some mediator particles with some rules governing which vertexes may be constructed. We will allow the Feynman Diagrams to obfuscate the computations, and focus on a qualitative description of these processes.

The Standard Model describes how matter particles, described above, interact with each other through the exchange of force mediating particles. Let us consider a simple scenario of electron electron interaction, which happens through the exchange of the electromagnetic mediating particle known as the photon. This process can be drawn using a vertex, as can be seen in Figure \ref{Fig:Intro:Vertex1}. 
\begin{figure}[h]
    \centering
        \includegraphics[width=0.33\textwidth]{F1/Vertex1}
        \caption{Visual representation of the interaction between two electrons and a photon.}
        \label{Fig:Intro:Vertex1}
\end{figure}
This vertex can be arranged and combined with other vertices in a variety of ways to represent different interactions between electrons, positrons, and photons. For example, we can consider the case where two electrons are propelled towards each other. Using two of the vertex pictured in \ref{Fig:Intro:Vertex1}, we can see that this must result in two electrons, as can be seen in Figure \ref{Fig:Intro:Feynman1}, where time propagates to the right on the x-axis, and distance is represented on the y-axis. 
\begin{figure}[h]
    \centering
        \includegraphics[width=0.49\textwidth]{F1/FeynDiag1}
        \caption{A Feynman Diagram depicting the interaction of two electrons. The action of the electromagnetic force is equivalent to the exchange of a photon. The vertical axis represents the distance between the electrons and the horizontal axis the passage of time.}
        \label{Fig:Intro:Feynman1}
\end{figure}
Alternatively, if we were to consider the interaction between an electron an a positron, which can be represented as an electron with an arrow pointing backwards in time, we can see that the resulting process is described by Figure {Fig:Intro:Feynman2}, where an electron and a positron annihilate to a photon, which produces anelectron positron pair.
\begin{figure}[h]
    \centering
        \includegraphics[width=0.49\textwidth]{F1/FeynDiag2}
        \caption{A Feynman Diagram depicting the annihilation of a positron and an electron.}
        \label{Fig:Intro:Feynman2}
\end{figure}
In both of these diagrams, we notice that charge is conserved on either side of the diagram. Many more diagrams can be made with this vertex, combining it with other vertices that involve a photon as well. In understanding these diagrams, we can explain three of the four fundamental forces at a qualitative level, below, while also introducing new theories in later chapters that have motivated the search for new particles.

\subsection{The Electromagnetic Force:}
Quantum electrodynamics (QED) describes how the electromagnetic force behaves at the quantum level. The force carrier of QED is the photon, $\gamma$, which is massless and moves at the speed of light. This theory can be described fully by Figure \ref{Fig:Intro:Vertex1}, since the photon has neutral electric charge and cannot couple to itself. Many experiments have performed precision measurements of QED properties, verifying that they match SM predictions extremely well.

\subsection{The Strong Force:}
Quantum Chromodynamics (QCD) describes quantum interactions related to the strong force. The force carrier of QCD is the gluon, g, which is massless and moves at the speed of light. The gluon does have color charge, just as the quarks do, and therefore there are a number of vertices in QCD, as shown in Figure \ref{Fig:Intro:Vertex2}.
\begin{figure}[h]
    \centering
        \includegraphics[width=\textwidth]{F1/Vertex2}
        \caption{Three vertexes arising from QCD interactions. The curly line represents gluons. Note that gluons, which are colored, may interact with themselves.}
        \label{Fig:Intro:Vertex2}
\end{figure}
As mentioned before, each quark has a color charge, and each anti-quark has an anti-color charge. Since color must be conserved, we can see that the gluon must carry a color and an anti-color for the vertices in Figure \ref{Fig:Intro:Vertex2} to be true. Naively, since there are three colors, one would think there are nine gluons, or nine possible combination of three colors and three anticolors. These are summarized below:

{\centering

$\frac{r\overline{r} + b\overline{b} + g\overline{g}}{\sqrt{3}}$

}

{\centering

    $\frac{r\overline{b} + b\overline{r}}{\sqrt{2}}$\hspace{1cm}$\frac{r\overline{g} + g\overline{r}}{\sqrt{2}}$\hspace{1cm}$\frac{b\overline{g} + g\overline{b}}{\sqrt{2}}$\hspace{1cm} $\frac{r\overline{r} - b\overline{b}}{\sqrt{2}}$
    
    }

{\centering  

    $\frac{-i(r\overline{b} - b\overline{r})}{\sqrt{2}}$\hspace{1cm}$\frac{-i(r\overline{g} - g\overline{r})}{\sqrt{2}}$\hspace{1cm}$\frac{-i(b\overline{g} - g\overline{b})}{\sqrt{2}}$ \hspace{1cm}$\frac{r\overline{r} + b\overline{b} -2g\overline{g}}{\sqrt{6}}$
    
    }

Particles comprised of quarks must be color neutral, both mesons (two-quark bound states) and baryons (three-quark bound states). We know that a particle can be neutral if it contains either quarks with color anti-color, or all three color quarks. Then since the combination of red, green, and blue is neutral, the first combination listed would be color neutral. However, color neutral particles must be non-interacting, otherwise colorless baryons would emit these gluons and interact with one another through the strong force, which we do not observe in nature. Therefore, this first combination is not possible, and there are only eight gluons, called the color octet.

The strong force is responsible for hadronization, or the production of many quarks and gluons when quarks are smashed apart due to an event such as a proton-proton collision. As quarks drift apart after the collision, a color tube of self-interacting gluons is created between the quarks. These tubes are stretched as the quarks drift further apart, increasing energy in the tube due to the constant force exerted from stretching. At distances of roughly $10^{-15}$ m, it becomes more energetically favorable for two new quarks to be created from the vacuum, through the process shown in the top left vertex in Figure \ref{Fig:Intro:Vertex2}. If these new quarks are still too energetic to be contained in a particle, this process will repeat until the energy has been sufficiently decreased for all new quark pairs to stay bound. This property is known as confinement, and is critical to the understanding of hadronic activity within a particle detector.

%ALICE START HERE
\subsection{The Weak Force:}
In this very distant past, the Weak force and the Electromagnetic force were one. They were called the Electroweak Interaction, and there were four carrier particles, $W_1$, $W_2$, $W_3$ and $B$. All four of these were massless, and had unit spins (as such, they were \textit{bosons}). While temperatures were high enough, these bosons were able to ignore the effects of the Higgs field, which we discuss in the next section.

As the universe cools down, the four bosons begin to interact with this field. The fermions now no longer interact directly with the $W_1$, $W_2$, $W_3$ and $B$, but rather, with superpositions of them. Where previously there were a $B$ and a $W_3$ Boson, we now have, $\gamma$ and $Z$. This superposition is simple to write:
\begin{equation}
    \left( \begin{array}{c} \gamma \\ Z \end{array} \right) = \left( \begin{array}{c} \cos\theta_W \\ -\sin\theta_W \end{array} \begin{array}{c} \sin\theta_W \\ \cos\theta_W \end{array} \right) \left( \begin{array}{c} B \\ W_3 \end{array} \right)
\end{equation}
Where $\theta_W$ called the Mixing Angle or Weinberg Angle, is a parameter of the Standard Model. The photons of the Electromagnetic force are just a particular combination of the $B$ and $W_3$. The $Z$ Boson is one of the carrier particles of the Weak Force. Similarly ,the $W_1$ and $W_2$ bosons combine via the superposition
\begin{equation}
    W^\pm = \frac{(W_1\mp iW_2)}{\sqrt{2}}
\end{equation}
to form two new Bosons, the $W^+$ and $W^-$. 

We've already talked about the behavior of the photon, so we just need to detail the vertexes involving the W and Z bosons for a full description of the Weak Force. These vertexes are shown in Figure \ref{Fig:Intro:Vertex3}.
\begin{figure}[h]
    \centering
        \includegraphics[width=\textwidth]{F1/Vertex3}
        \caption{Vertexes arising from weak interactions. The wavy lines represent the Bosons. The straight lines represent any particle where the appropriate quantities are conserved (details in the text).}
        \label{Fig:Intro:Vertex3}
\end{figure}
The Z Boson is not massless, it has mass $91.2$ GeV/c$^2$\footnote{The unit GeV/c$^2$ is a unit of mass. One eV/c$^2$ is equal to $1.78 \times 10^{-36}$ kg. An electron has mass $\sim$ 0.5 MeV/c$^2$ and a proton $\sim$ 1 GeV/c$^2$.}. Z bosons couples particles to their anti-particles. For example, the process $Z\rightarrow e^-e^+$ is allowed but the process $Z\rightarrow\mu^+e^-$ is not. Similarly, $Z\rightarrow b\overline{b}$ is allowed but $Z\rightarrow b\overline{u}$ is not. In this regard, it behaves like a heavy version of the photon, except that it can couple to neutral particles (for example, $Z\rightarrow\nu_e\overline{\nu}_e$).

The W boson (with a mass of $80.4$ GeV/c$^2$) also couples pairs of fermions, except that it links the flavors within a generation. For example, a W boson might decay to two quarks ($W\rightarrow u\overline{d}$) or to a lepton and its neutrino ($W\rightarrow\mu\nu_\mu$). It is charged, so the process $\gamma\rightarrow W^+W^-$ is allowed. The decay $Z\rightarrow W^+W^-$ is also allowed.

The W boson is a fascinating object to study. As we've already mentioned in passing: The weak interaction only acts on left-handed particles and right-handed anti-particles\footnote{This causes considerable complications because neutrinos have mass. It is therefore possible to define a reference frame in which the neutrino is moving backwards with relation to the W it decayed from, thus flipping its helicity...}.

The decay $\gamma\rightarrow \mu^+e^-$ is a perfectly plausible decay, seeing as charge is conserved. However, it never occurs. Similarly, the Z boson might be allowed to decay to different generation quarks (this wouldn't violate charge or color), but it simply does not happen. These decays are called \textit{flavor changing neutral currents}. Flavor changing \textit{charged} currents on the other hand, do exist. Specifically, Ws can decay to quark pairs outside of their generation. For example, we might see $W\rightarrow s\overline{u}$. The probability of a W decaying to different generations is encoded in the CKM Matrix, an empirically measured set of values. Here is that matrix:
\begin{equation}\label{Eq:CKM}
    \left(\begin{array}{c} V_{ud}\\V_{cd}\\V_{td} \end{array}\begin{array}{c}  V_{us}\\V_{cs}\\V_{ts} \end{array}\begin{array}{c} V_{ub}\\V_{cb}\\V_{tb}  \end{array}\right) = \left(\begin{array}{c} 0.974\\0.225\\0.009 \end{array}\begin{array}{c}  0.225\\0.973\\0.040 \end{array}\begin{array}{c} 0.004\\0.041\\0.999  \end{array}\right)
\end{equation}
where $|V_{ij}|^2$ is the probability that a quark $i$ decays to a quark $j$ through the emission of a W.

It is then because of the W and the W only that our world seems to be composed of only up and down quarks and electrons. All other particles will eventually decay (through a W) to something lighter. The other leptons (the muons and the taus) will eventually decay by emitting a neutrino and a W, with the W decaying to an electron and a neutrino. All the various particles discovered in the last century eventually decayed to protons and neutrons because the higher order quarks embedded in them decayed (again, through a W) back to the first generation.

\subsection{Mass and the Higgs Boson}
The Higgs Boson is the most recent addition to the Standard Model to be discovered. It is responsible for giving the other particles their mass\footnote{except the neutrinos, which by now you must have noticed, are devious...} As we already mentioned, this does not happen at very high energies. To understand this, we should consider the Higgs Potential. For the purpose of illustration we can simplify it to the form $\sim (\phi^2 - \eta^2)^2$, where $\phi$ is the Higgs Field, which we plot in Figure \ref{Fig:Intro:HiggsPot}\footnote{This shape is sometimes called the ``mexican hat'' or ``champagne bottle'' potential, depending on the level of cultural sensitivity required.}. 
\begin{figure}[h]
    \centering
        \includegraphics[width=0.66\textwidth]{F1/higsspot}
        \caption{Sketch of the Higgs Potential. Notice that while the over-all shape is symmetric about the y-axis, the function is not symmetric about the minima.}
        \label{Fig:Intro:HiggsPot}
\end{figure}
The potential in question is symmetric (about the y-axis, which in the case of the actual Higgs potential would have units of energy). In the high-temperature early universe, the energies of particles was such that the "bump" at the bottom of the potential played no part. As the universe cools, however, things must settle into either of the two valleys. It's not the case that the equations of of the Standard Model aren't symmetric, it's that practically, at the energies in question, they cause behavior which is not. This is called \textit{spontaneous symmetry breaking}.

This asymmetry causes the Higgs field in the low temperature universe to take on a vacuum expectation value (roughly speaking: its average value in empty space) which is non-zero (it's 246 GeV). We usually abbreviate this as just the \textit{VEV}. The VEV couples to the electroweak interactions, creating the photon and the weak force bosons as we experience them.

Through a similar but not identical process, this spontaneously broken Higgs field also couples to the weak bosons, quarks, electron, muon and tau, giving them mass where without it they were massless. The neutrinos do not couple, and are therefore massless in the standard model.

So far we've been talking about the Higgs field, but as you may have heard, there is also a Higgs Boson. This particle is just an excitation of the Higgs Field the way we think of a photon as being an excitation of the electromagnetic field. It was discovered in 2012 at the Large Hadron Collider after a 40 year long search. This Higgs has no charge, has no spin, and has a mass of 126 GeV. It couples to anything with mass, which means it couples to itself. The Higgs Vertex is the very last piece of the Standard Model, and is shown in Figure \ref{Fig:Intro:Vertex4}.
\begin{figure}[h]
    \centering
        \includegraphics[width=0.5\textwidth]{F1/Vertex4}
        \caption{The Higgs Vertex. The dashed line is the Higgs Boson and the straight lines represent any massive particle (including the Higgs).}
        \label{Fig:Intro:Vertex4}
\end{figure}
\section{Rounding Up the Standard Model Particles}
So to recap, there are, once we count all the colors and anti-particles, 61 particles in the standard model. We know they interact, and we can pictorally express a number of complex interactions using the Feynman diagram. For example, consider the diagram in Figure \ref{Fig:Intro:ttbar}.
\begin{figure}[h]
    \centering
        \includegraphics[width=0.5\textwidth]{F1/ttbarIntroDiag}
        \caption{Collider Example: two quarks collide and form a gluon, which decays to two top quarks.}
        \label{Fig:Intro:ttbar}
\end{figure}
This is an important physics process\footnote{incidentally, the primary background of this Thesis' analysis} found in high energy colliders, and we'll discuss it in much more detail later on. For now, it serves as a useful example of the kind of processes we can construct with our Feynman diagrams: two protons collide and quarks inside of them annihilate\footnote{See Chapter \ref{Sec:CMS} for a more formal description of what's happening there} into a gluon. That gluon propagates some distance before it decays back to quarks, this time to a pair of top quarks. Each of these decays through Weak interactions to a b and a W (this is almost always the case, see Equation \ref{Eq:CKM}). One W decays to a lepton and a neutrino, and the other to a pair of quarks. This diagram then contains five distinct vertexes.

Indeed, we can make these diagrams as complicated as we want. Some of them become quite silly. See Figure \ref{Fig:Intro:Diags} for some more entertaining examples.\begin{figure}[h]
    \centering
        \includegraphics[width=0.4\textwidth]{F1/goof3}
        \includegraphics[width=0.4\textwidth]{F1/goof2}
        \includegraphics[width=0.66\textwidth]{F1/goof1}
        \caption{Some more Feynman Diagrams representing real physics processes. Some of these play real roles in modern analyses and some of these were constructed for the sake of the diagram itself.}
        \label{Fig:Intro:Diags}
\end{figure}

We've given a mostly qualitative description of the model, without differentiating between facts gleaned from experiments and facts that come from calculations within the model itself. Fortunately, the Standard Model is extremely self-consistent and despite there being 61 particles, 4 forces and 1 Higgs field, there are only 19 ``free parameters'' which have to be determined experimentally. Most of them are masses. The Higgs Boson is the reason that the particles have mass, but it doesn't specify what those masses should be. What was the question ``why does a particle have this mass'' now becomes ``why is this particle's coupling to the Higgs what it is''. The CKM Matrix's nine entries can be interrelated with just four parameters. The remaining parameters have to do with the Higgs field or the couplings of the forces.

\subsection{The top quark}
No discussion of the standard model is complete unless it mentions how strange the top quark is\footnote{even though it has strangeness 0.}. The top quark's mass is 174 GeV/c$^2$. The next heaviest quark barely makes it to the 5 GeV/c$^2$ mark, with all other fermions being lighter than it. 

This might just be a curiosity, as there is nothing inherently ``wrong'' with a very heavy top, but it does lead to some strange behavior. The top quark is the only quark that can exist outside of a color singlet. This is because the color tubes we mentioned before never form: due to its very large mass, the top decays immediately (in $5\times 10^{-25}$ seconds) through the Weak force to a bottom quark (or very rarely, a c or a d) and a W Boson. 

This does not mean that the top quark can only interact weakly, just that it decays that way. The top quark plays an important role in Higgs production (since the Higgs couples so strongly to it), as shown in Figure \ref{Fig:Intro:TopDecayHiggs}. As we will see, the top quark is a frequent player in searches for new physics as new theories often couple preferentially to massive objects.
\begin{figure}[h]
    \centering
        \includegraphics[width=0.66\textwidth]{F1/higgsProd}
        \caption{Production of a Higgs boson from the collision of two gluons (which, notably, cannot couple directly to the Higgs). The triangular structure is a \textit{loop}.}
        \label{Fig:Intro:TopDecayHiggs}
\end{figure}
\section{A Few Open Questions}
For a supposed Theory of Everything, the Standard Model could use some improvement. It only covers a slim $4.6\%$ of the content of the known universe. Dark Matter, something completely outside the Standard Model, accounts for a further $24\%$, with the rest being mysterious Dark Energy. Both these Dark Entities loom as considerable proof that while the Standard Model does a good job of describing that little $4.6\%$ we call \textit{Baryonic Matter}, it is far from a complete description of the world.

Actually, the Standard Model isn't even a complete description of Baryonic Matter. The obvious problem being of course, that it fails to explain, or in any way even include, the most obvious of forces: Gravity. Gravity is not included in the Standard Model. It's easy to just say that gravity just doesn't matter at the very small distances the Standard Model applies to, but the addition of some massless force-carrying \textit{Graviton} would certainly be welcome.

Of course, there's also these pesky neutrinos and their unwarranted masses to directly (experimentally) challenge our model. These are not the only hints that we're missing something. One obvious observation is that while the Standard Model treats positrons and electrons (and quarks and anti-quarks) symmetrically, the universe's Baryonic Matter is overwhelmingly composed of electrons and quarks. What happened to all the anti-particles?

And what are these insanely energetic cosmic rays we detect from time to time? The Standard Model puts limits on the energy a cosmic ray can attain before it is slowed down by interactions in space... and yet, we have detected particles some orders of magnitude above that limits.

One of the great strengths of the Standard Model is that it is incredibly self-consistent. Indeed, it has been tested and re-tested, with no obvious internal flaw. Now however, as we start to seek the answer to new questions, this strength becomes a flaw as there is no way to modify the model without un-hinging some other internal aspect. To answer the questions of modern physics then, we must venture into new territory Beyond The Standard Model!