\chapter{Semi-resolved Analysis}\label{Sec:Ana}

This thesis is focused on events at the LHC with two Higgs bosons that each decay to two b quarks, or HH$\rightarrow$bbbb. We are looking for resonant production coming from a particle such as the bulk graviton or radion, and non-resonant production through SM and perhaps other BSM diagrams. However, these processes are much less common than QCD multijet processes at the LHC. In order to effectively probe this new physics, we must reduce as much background as possible, while retaining as much signal as possible.

We begin the process of optimizing the ratio of signal to background by considering important characteristics of our signal events. Recalling the event depiction, shown in Figure~\ref{Fig:threecases2again}, we can identify a number of variables that may help with the elimination of background.
\begin{figure}[h!]
    \centering
        \includegraphics[width=0.7\textwidth]{F4/semiresolved.png}
        \caption{Signature for semi-resolved two Higgs to four b-quarks.}
        \label{Fig:threecases2again}
\end{figure}
First, we focus on the topology of the event. Signal events should have one large AK8 jet, representing the boosted Higgs. In addition, there should be two AK4 jets that are outside the cone of the AK8 jet, and close to each other. These two AK4 jets representing the resolved Higgs. We expect a high amount of hadronic activity in each event. While the process of hadronization can (and often does) produce leptons, we expect that these electrons or muons would not be isolated from the hadronic activity, and would be low momentum. We expect that the combination of the two AK4 jets should be roughly opposite that of the AK8 jet, which is to say they should have similar pseudorapidity. 

Now, we break it down, side by side. We expect that the AK8 jet will have a high momentum, since it had enough momentum for the Higgs decay products to be collimated into one jet. We expect that this jet will not be too close to the beamline, or the pseudorapidity should be low. We expect that the softdrop mass of this jet should be close to the Higgs mass. The $\tau_{21}$ of the jet should be close to 0, since we expect this jet to be more likely to have two subjets than no substructure. This jet should have a double-b tagger value close to 1, indicating that it is likely to have two b-quarks inside the jet.

For the AK4 jet side, we expect that these jets should not have low momentum. Since they are the individual b-jets rather than the combined Higgs jet, and the Higgs did not have enough momentum to collimate these two jets into one big jet, we do not necessarily expect that these jets would have a momentum as large as the AK8 jet, but the momentum of these two jets should still be higher than that of typical QCD jets. We also expect that these jets will not be too close to the beamline, or the pseudorapidity should be low. If we add up the four-vectors of the two AK4 jets, the mass of this combination should be close to that of the Higgs. Each AK4 jet should have a deepCSV value close to 1, since we expect both jets to be the result of a b-quark.

All of these considerations should help eliminate QCD multijet background, while retaining most of our signal events. However, this is not the only background we must consider. It turns out that top-anti-top (tt) can also contribute to the background. Top quarks decay into a W boson and a b-quark, so events with two tops contain two W's and two b-jets. While W bosons can decay leptonically, they decay hadronically 2/3 of the time. Because of the high mass of the top quark (even higher than the Higgs at 172 GeV), these events can mimic HH$\rightarrow$bbbb events more easily than QCD multijet processes can. Luckily, the rate at which tt events are produced is much lower than that of QCD multijets. Still, we must take into consideration how to eliminate as much tt as possible in addition to eliminating QCD multijet background. 

Data comes out of the LHC as information from various detectors, and we can only infer what the actual events may have been. In order to better understand the data we see, we use Monte Carlo simulations of signal and background processes to study different properties of signal and background events. Monte Carlo (MC) simulations are generated by random sampling, where statistical information on what could happen in a given event informs what is in each MC event. Once MC events have been produced, the events are sent through a simulation of the CMS detector in order to mimic data taken by CMS. We then can compare signal and background MC distributions of many different variables to figure out how to prune the data to get the best signal to background ratio possible.

In the following sections, we will cover the data and MC used as well as the event selection of this analysis, detailing each choice and demonstrating its efficacy through looking at these variables in MC simulation of signal and background.

\section{Data samples\label{ss:DataSamples}}

The analysis is performed using pp interactions collected with the CMS detector at $\sqrt{s}=13$ TeV.
%, as measured using the golden JSON file \href{https://cms-service-dqm.web.cern.ch/cms-service-dqm/CAF/certification/Collisions16/13TeV/ReReco/Final/Cert_271036-284044_13TeV_23Sep2016ReReco_Collisions16_JSON.txt}{Cert\_271036-284044\_13TeV\_23Sep2016ReReco\_Collisions16\_JSON.txt}. 
The data samples are summarized in Table~\ref{tab:data}. The JetHT dataset was chosen due to the high amount of hadronic activity in these events. Each run corresponds to a different time period in which data was taken, and the total dataset corresponds to an integrated luminosity, or total amount of data, of 35.9 $\text{fb}^{-1}$.

\begin{table}[htb]
  \begin{center}
    \caption{List of primary datasets for the pp collisions at $\sqrt{s} = 13$ TeV and their corresponding integrated luminosities.} 
    %The golden JSON file \href{https://cms-service-dqm.web.cern.ch/cms-service-dqm/CAF/certification/Collisions16/13TeV/ReReco/Final/Cert_271036-284044_13TeV_23Sep2016ReReco_Collisions16_JSON.txt}{Cert\_271036-284044\_13TeV\_23Sep2016ReReco\_Collisions16\_JSON.txt} was used.}
    \begin{tabular}{l|c|c}
      \hline
      \hline
      Dataset & Processing & Int. lumi. (fb$^{-1}$) \\
      \hline
      JetHT/Run2016B   & 03Feb2017 & 5.9  \\
      JetHT/Run2016C   & 03Feb2017 & 2.6  \\
      JetHT/Run2016D   & 03Feb2017 & 4.4  \\
      JetHT/Run2016E   & 03Feb2017 & 4.1  \\
      JetHT/Run2016F   & 03Feb2017 & 3.2  \\
      JetHT/Run2016G   & 03Feb2017 & 7.7  \\ 
      JetHT/Run2016H   & 03Feb2017 & 8.9 \\ 
      \hline
      Total & & 35.9 $\text{fb}^{-1}$ \\ 
      \hline
      \hline  
    \end{tabular}  
    \label{tab:data}
  \end{center}
\end{table}

\section{MC simulation\label{ss:MCSimulation}}

The signal MC samples used for this analysis includes spin-0 bulk graviton and spin-2 radion resonances which both decay to HH$\rightarrow$bbbb, as well as non-resonant HH$\rightarrow$bbbb, 
given in Table~\ref{tab:signal_MC}. We examine a range of different masses for the bulk graviton and radion resonances, chosen such that we would expect at least some events to present as semi-resolved, rather than fully resolved or fully boosted. We also examine all of the different non-resonant BSM benchmarks, as well as the SM non-resonant production.

\begin{table}[htb]
  \begin{center}
    \caption{List of signal MC samples used.\label{tab:signal_MC}}
    \begin{tabular}{|l|c|c}
      \hline
      \textbf{Resonant Bulk graviton} \\
      \hline
      {GluGluToBulkGravitonToHHTo4B\_M-500\_narrow\_13TeV-madgraph}  \\
      {GluGluToBulkGravitonToHHTo4B\_M-550\_narrow\_13TeV-madgraph} \\
      {GluGluToBulkGravitonToHHTo4B\_M-600\_narrow\_13TeV-madgraph} \\
      {GluGluToBulkGravitonToHHTo4B\_M-650\_narrow\_13TeV-madgraph}  \\
      {GluGluToBulkGravitonToHHTo4B\_M-750\_narrow\_13TeV-madgraph} \\
      {GluGluToBulkGravitonToHHTo4B\_M-800\_narrow\_13TeV-madgraph} \\
      {GluGluToBulkGravitonToHHTo4B\_M-900\_narrow\_13TeV-madgraph} \\
      {BulkGravTohhTohbbhbb\_narrow\_M-1000\_13TeV-madgraph}  \\
      {BulkGravTohhTohbbhbb\_narrow\_M-1200\_13TeV-madgraph}  \\
      {BulkGravTohhTohbbhbb\_narrow\_M-1400\_13TeV-madgraph}  \\
      {BulkGravTohhTohbbhbb\_narrow\_M-1600\_13TeV-madgraph} \\
      {BulkGravTohhTohbbhbb\_narrow\_M-1800\_13TeV-madgraph}\\
      {BulkGravTohhTohbbhbb\_narrow\_M-2000\_13TeV-madgraph} \\
      {BulkGravTohhTohbbhbb\_narrow\_M-2500\_13TeV-madgraph} \\
      {BulkGravTohhTohbbhbb\_narrow\_M-3000\_13TeV-madgraph} \\
      \hline
      \textbf{Radion} \\ 
      \hline
      {GluGluToRadionToHHTo4B\_M-600\_narrow\_13TeV-madgraph}  \\
      {GluGluToRadionToHHTo4B\_M-650\_narrow\_13TeV-madgraph} \\
      {GluGluToRadionToHHTo4B\_M-750\_narrow\_13TeV-madgraph} \\
      {GluGluToRadionToHHTo4B\_M-800\_narrow\_13TeV-madgraph}  \\
      {RadionTohhTohbbhbb\_narrow\_M-1000\_13TeV-madgraph} \\
      {RadionTohhTohbbhbb\_narrow\_M-1200\_13TeV-madgraph} \\
      {RadionTohhTohbbhbb\_narrow\_M-1400\_13TeV-madgraph} \\
      {RadionTohhTohbbhbb\_narrow\_M-1600\_13TeV-madgraph}  \\
      \hline
      \textbf{Non-resonant Signal} \\ 
      \hline
      GF\_HHTo4B\_node10\_13TeV-madgraph-pythia8\\
      GF\_HHTo4B\_node11\_13TeV-madgraph-pythia8\\
      GF\_HHTo4B\_node12\_13TeV-madgraph-pythia8\\
      GF\_HHTo4B\_node1\_13TeV-madgraph-pythia8\\
      GF\_HHTo4B\_node2\_13TeV-madgraph-pythia8\\
      GF\_HHTo4B\_node3\_13TeV-madgraph-pythia8\\
      GF\_HHTo4B\_node4\_13TeV-madgraph-pythia8\\
      GF\_HHTo4B\_node5\_13TeV-madgraph-pythia8\\
      GF\_HHTo4B\_node6\_13TeV-madgraph-pythia8\\
      GF\_HHTo4B\_node7\_13TeV-madgraph-pythia8\\
      GF\_HHTo4B\_node8\_13TeV-madgraph-pythia8\\
      GF\_HHTo4B\_node9\_13TeV-madgraph-pythia8\\
      GluGluToHHTo4B\_node\_SM\_13TeV-madgraph  \\
 %     \hline
 %     \multicolumn{3}{c}{Non-resonant Bulk graviton} \\ \cline{1-3}
 %     \hline
 %         {GluGluToHHTo4B\_node\_10\_13TeV-madgraph} & -- & 300000 \\
 %         {GluGluToHHTo4B\_node\_11\_13TeV-madgraph} & -- & 300000 \\
 %         {GluGluToHHTo4B\_node\_12\_13TeV-madgraph} & -- & 300000 \\
 %         {GluGluToHHTo4B\_node\_13\_13TeV-madgraph} & -- & 300000 \\
 %         {GluGluToHHTo4B\_node\_2\_13TeV-madgraph} & -- & 300000 \\
 %         {GluGluToHHTo4B\_node\_3\_13TeV-madgraph} & -- & 300000 \\
 %         {GluGluToHHTo4B\_node\_4\_13TeV-madgraph} & -- & 300000 \\
 %         {GluGluToHHTo4B\_node\_5\_13TeV-madgraph} & -- & 300000 \\
%          {GluGluToHHTo4B\_node\_6\_13TeV-madgraph} & -- & 300000 \\
%          {GluGluToHHTo4B\_node\_7\_13TeV-madgraph} & -- & 300000 \\
%          {GluGluToHHTo4B\_node\_8\_13TeV-madgraph} & -- & 300000 \\
%          {GluGluToHHTo4B\_node\_9\_13TeV-madgraph} & -- & 300000 \\
%          {GluGluToHHTo4B\_node\_SM\_13TeV-madgraph}  & -- & 300000 \\
%          {GluGluToHHTo4B\_node\_box\_13TeV-madgraph} & -- & 300000 \\
      \hline
    \end{tabular}
  \end{center}
\end{table}

We also consider different background processes that are likely to contribute to the analysis. The two largest are QCD multijet and tt MC, as given in Table~\ref{tab:bkg_MC}. The QCD samples are separated into $H_{T}$ ranges, where $H_{T}$ is the summed $p_{T}$ of all jets in the event. Other MC samples related to diboson production were examined, but found to have no appreciable impact.

\begin{table}[htb]
  \begin{center}
    \caption{List of background Monte Carlo samples used. }
    \label{tab:bkg_MC}
    %The two \ttjets \textsc{Powheg} samples correspond to two different productions with the same generator parameters, but the latter with a much higher statistics. The cross sections $\sigma$ and number of events generated are also given. The cross sections for the QCD processes are at LO and are taken from McM. The other SM background cross sections are taken from Ref.~\cite{SMXsecTWiki13TeV}.\label{tab:bkg_MC}}
    \begin{tabular}{|l|c|c}
      \hline
      Background \\ 
      \hline
      {QCD\_HT-100to200}   \\
      {QCD\_HT-200to300}  \\
      {QCD\_HT-300to500}\\
      {QCD\_HT-500to700}   \\
      {QCD\_HT-700to1000} \\
      {QCD\_HT-1000to1500}  \\
      {QCD\_HT-1500to2000} \\
      {QCD\_HT-2000toinf}   \\
      \hline
      {TT\_TuneCUETP8M1\_13TeV-powheg-pythia8} \\
      \hline
    \end{tabular}
  \end{center}
\end{table}

%In addition to accounting for the MC sample's behavior inside the detector, pileup is also accounted for. The Moriond2017 pileup scenario was used to approximate the number of inelastic collisions per bunch crossing at LHC 13~\TeV data-taking using an inelastic $\Pp\Pp$ collision cross section of 69.2\unit{mb}~\cite{Aaboud:2016mmw}.
\section{Event selection}

Events are required to have at least one reconstructed pp collision vertex.
Many additional vertices, corresponding to other overlapping pp collisions (pileup), are usually reconstructed in an event using charged particle tracks. We assume that the primary interaction vertex (PV) corresponds to the one that maximizes the sum in $p_{T}^2$ and the magnitude of $\sum{p_{T}}$ from the associated physics objects. 

\subsection{Jet kinematics selection\label{ss:JetSel}}

Individual particles are reconstructed using the PF algorithm described previously. The five classes of PF candidates are  muons, electrons, photons, charged hadrons, and neutral hadrons. 

This analysis used AK8 jets and AK4 jets.
In order to mitigate the effect of pileup on jet observables, we take advantage of pileup per particle identification (PUPPI) ~\cite{puppi} for AK8 jets. This method uses local shape information, event pileup properties and tracking information together in order to compute a weight describing the degree to which a particle is pileup-like. No additional pileup corrections are applied to AK8 jets clustered from these weighted inputs.

The jet 4-momenta are corrected to account for the difference between the measured and the expected momentum at the particle level, using a standard CMS correction procedure described in Refs.~\cite{JINST6,CMS-DP-2013-011}. The \texttt{Summer16\_23Sep2016V4} jet energy corrections~\cite{JESUncertaintyTWiki} were used. The \texttt{Spring16\_25nsV10} jet energy resolutions are used.  All AK8 jets are further required to pass TightLepVeto jet identification requirements~\cite{JetID13TeVTWiki}, which was chosen over Tight to reject leptons more efficiently. These requirements ensure that the jet is comprised of mostly charged hadrons, with more than one constituent, and limits the amount of neutral hadrons, electrons, and muons in the jet.
%\begin{table}[htb]
%  \begin{center}
%    \topcaption[The ``TightLepVeto'' PF jet identification quality criteria used in the analyses.]{TightLepVeto PF jet identification quality criteria used in the analyses.\label{tab:tightjetid}}
%    \begin{tabular}{l r}
 %   \hline
 %   \hline
 %   Variable &  Cut \\
 %   \hline
 %   Neutral Hadron Fraction & $<0.90$  \\
 %   Neutral EM Fraction     & $<0.90$  \\
%    Number of Constituents  & $>1$     \\
%    Muon Fraction           & $<0.8$   \\
 %   Charged Hadron Fraction & $>0$     \\
 %   Charged Multiplicity    & $>0$     \\
 %   Charged EM Fraction     & $< 0.90$ \\
%    \hline
%    \hline
%    \end{tabular}
%  \end{center}
%\end{table}
Figs.~\ref{fig:AK8pteta},~\ref{fig:AK41pteta},~\ref{fig:AK42pteta} show the $p_{T}$ and $\eta$ of the three selected jets. A preselection is applied to QCD MC, tt MC, and bulk graviton signal points for masses 600, 800, 1000, and 1200 GeV as follows:
\begin{itemize}
 \item Passes trigger selection (discussed later in this chapter);
 \item At least one AK8 jet in the event with $p_{T} > 300\GeV$ and $|\eta| < 2.4$;
 \item At least two AK4 jets with $p_{T} > 30\GeV$ and $|\eta| < 2.4$, with a deep CSV value $>$ 0.2219;
 \item The AK8 jet and AK4 jets that represent the three signal jets are chosen in the following way. First we pick out AK4 jets that have $p_{T}$ $>$ 30, $|\eta| <$ 2.4, and deep CSV $>$ 0.6324. Out of this group we find all AK4 jet pairs where each AK4 jet is at least $\Delta R >$ 0.8 away from the highest $p_{T}$ AK8 jet, and within $\Delta R$ of 1.5 of each other. If there are more than one pair, we pick the pair with the highest deepCSV values. If there are no pairs, we repeat this second step for the second highest $p_{T}$ AK8 jet and check if we get a pair of AK4 jets that match the criteria. Once again, if there are more than one pair, we pick the pair with the highest deepCSV values, and if there are none, the event is rejected. 
 %\item Out of the AK8 and AK4 jets that satisfy the above requirements, one AK8 jet and two AK4 jets are chosen to be the three signal jets. The highest $p_{T}$ AK8 jet with $M_{\rm soft\,drop} > 40$ \GeV that is $\Delta R$ $>$ 0.8 away from two AK4 jets that are within $\Delta R$ $<$ 1.5 of each other is chosen. If there are more than two AK4 jets that satisfy this criteria, the two AK4 jets with the highest deepCSV value are chosen.
% \item The AK8 jet is $\Delta R$ $>$ 0.8 away from each AK4 jet, and the two AK4 jets are within $\Delta R$ $<$ 1.5,
 %      and if more than one AK8 + two AK4 triplet is found, the triplet with highest $p_{T}$ AK8 jet and then the highest AK4 deep CSV values is chosen;
% \item Reduced mass, introduced in Eqn.~\ref{eq:Mjjs}, $>$ 700 GeV, since the analysis has little sensitivity below this value;
% \item $\Delta\eta$ = $|AK8\eta - (AK4jet1 + AK4jet2)\eta| < 2.0$;
%and if more than one AK8 + two AK4 triplet is found, the triplet with the |lowest M_{\rm soft\,drop} - 120| and the highest AK4 deep CSV values is chosen.
 \item Lepton veto (discussed later on);
  \item Fails selection of the fully boosted and the selection of the fully resolved analysis, which is documented in Sec.~\ref{ss:boostres}.
\end{itemize}

All figures in this section have the same preselection, with QCD MC (yellow) and tt (red) weighted by their cross section multiplied by the luminosity of the dataset and divided by the total number of events. The signal samples (different dashed lines) are weighted by 50 multiplied by the cross section multiplied by the luminosity of the dataset and divided by the total number of events, so that they are easy to see on the plots. 

\begin{figure}[thb!]
\begin{center}
\includegraphics[scale=0.34]{F5/shapeptFJ.pdf}
\includegraphics[scale=0.34]{F5/shapeetaFJ.pdf}\\
\end{center}
\caption{Left. The $p_{T}$ of the AK8 jet. Right. The $\eta$ of the AK8 jet.}
\label{fig:AK8pteta}
\end{figure} 

\begin{figure}[thb!]
\begin{center}
\includegraphics[scale=0.34]{F5/shapeptJ1.pdf}
\includegraphics[scale=0.34]{F5/shapeetaJ1.pdf}\\
\end{center}
\caption{Left. The $p_{T}$ of the highest $p_{T}$ selected AK4 jet. Right. The $\eta$ of the highest $p_{T}$ selected AK4 jet.}
\label{fig:AK41pteta}
\end{figure} 

\begin{figure}[thb!]
\begin{center}
\includegraphics[scale=0.34]{F5/shapeptJ2.pdf}
\includegraphics[scale=0.34]{F5/shapeetaJ2.pdf}\\
\end{center}
\caption{Left. The $p_{T}$ of the other selected AK4 jet. Right. The $\eta$ of the other selected AK4 jet.}
\label{fig:AK42pteta}
\end{figure} 

The analysis is performed in two different $\Delta\eta$ regions, 0--1.0 and 1.0--2.0, where $\Delta\eta = |\eta_{AK8} - \eta_{1AK4 + 2AK4}|$ where AK8 is the AK8 jet and the two AK4 jets are added together. This variable can be found in Figure~\ref{fig:deta}. 

\begin{figure}[thb!]
\begin{center}
\includegraphics[scale=0.34]{F5/shapedeta.pdf}
\end{center}
\caption{The absolute value of $\Delta\eta$ between the AK8 jet and the combined 4 vector of the AK4 jets. We require $|\Delta\eta| < 2.0$.}
\label{fig:deta}
\end{figure} 

As discussed previously, we expect a high $p_{T}$ AK8 jet, so we require AK8 jet $> 300$ GeV. The two AK4 jets are required to have $> 30$ GeV each, to prevent low $p_{T}$ QCD jets. All three jets are required to have $|\eta| < 2.4$, ensuring that the jets are not too close to the beamline. Lastly, we require $|\Delta\eta| < 2.0$ so that the AK8 jet and the result of the addition of the two AK4 jets have similar $\eta$, to ensure that they are in similar positions in the detector.

\subsection{\texorpdfstring{\PH}{H} mass selection\label{ss:EvtSelMass}}

The AK8 softdrop mass is also used to limit the amount of background. A dedicated jet energy calibration is applied to the softdrop mass as derived in Ref.~\cite{CMS-AN-16-235}. The correction is derived in two steps. First, a weight to account for a $p_{T}$ dependent softdrop jet mass shift introduced prior to MC being sent through the detector simulation is calculated. Second, to account for any residual $p_{T}$ and $\eta$ dependence, an additional weight is calculated based on the difference between the reconstructed (post-detector simulation) and the generated (pre-detector simulation) softdrop mass. The difference in reconstructed and generated softdrop mass is a 5-10\% effect. 
Figure~\ref{fig:AK8mass} shows the AK8 soft-drop corrected jet mass. 

The softdrop corrected mass, or $M_{\rm soft\,drop}$ signal mass window is restricted to between 105 and 135 \GeV, to avoid overlapping with other analyses targeting W and Z resonances. 

\begin{figure}[thb!]
\begin{center}
\includegraphics[scale=0.5]{F5/shapejetmass.pdf}
\end{center}
\caption{The soft-drop mass of the AK8 jet. We require it to be between 105 and 135 \GeV.}
\label{fig:AK8mass}
\end{figure} 

The combined mass of the two AK4 jets, defined as the mass of the two AK4 jet vectors added together, or the mass of (AK4jet1 + AK4jet2), can also be used to suppress the multijet and tt backgrounds as well, shown in Figure~\ref{fig:AK4dijetmass}. The AK4 dijet mass window is slightly larger, ranging from 90 to 140 \GeV, because there is less worry of an overlap with another analysis.

\begin{figure}[thb!]
\begin{center}
\includegraphics[scale=0.5]{F5/shapedijetmass.pdf}
\end{center}
\caption{The dijet mass of the AK4 jet. We require it to be between 90 and 140 \GeV.}
\label{fig:AK4dijetmass}
\end{figure}

\subsection{N-subjettiness selection\label{sec:EvtSelNsubjettiness}}

The ratio $\nsub = \tau_2/\tau_1$  is calculated for the AK8 jet after PUPPI has been applied. It is required to have $\nsub < 0.55$. 
The $\nsub$ spectra for the AK8 jet is shown in Figure~\ref{fig:tau21}.
%The efficiency of the $\nsub$ selection is found to be uniform over the whole mass range of the signal, as shown in Figure~\ref{fig:eff_comp_tau21}.

\begin{figure}[th!b]
\begin{center}
\includegraphics[scale=0.5]{F5/shapeptau21.pdf}
\end{center}
\caption{$\nsub$ distribution for the AK8 jet. We require $\tau_{21} <$ 0.55.\label{fig:tau21}}
\end{figure}

%\begin{figure}[th!b]
%\begin{center}
%\includegraphics[scale=0.34]{F5/signaleff_t21_NoHMass_PreDBT.pdf}
%\includegraphics[scale=0.34]{F5/signaleff_t21_not21_HMass_DBT.pdf}
%\end{center}
%\caption{Comparing the efficiency as a function of different bulk graviton signal masses: Of the $\nsub$ selection (without cut on the Higgs jet mass or the double b tagger) (left). The right figure compares the efficiency of the Higgs jet selection after the Higgs mass and the double b tagger selections, with and without the \nsub selection. The figure shows that the $\nsub$ selection has a uniform efficiency over the whole mass range.\label{fig:eff_comp_tau21}}
%\end{figure}

%\begin{figure}[h]
%\begin{center}
%\includegraphics[scale=0.34]{F5/ShapeComparison_hJet1bbtag.pdf}
%\includegraphics[scale=0.34]{F5/ShapeComparison_hJet2bbtag.pdf}
%\end{center}
%\caption{Double-b tagger discriminant for QCD background and different mass signals.}
%\label{fig:schema1}
%\end{figure}

\subsection{B-tagging}

In order to identify the AK8 jet most likely to contain two b quarks, we use the double-b tagger discriminant, shown in Figure~\ref{fig:doubleb}. We require double b-tagger $> 0.8$.

\begin{figure}[h]
\begin{center}
\includegraphics[scale=0.5]{F5/shapedoubleb.pdf}
\end{center}
\caption{Double-b tagger discriminant for the AK8 jet. We require double b tagger $>$ 0.8.}
\label{fig:doubleb}
\end{figure} 

For the AK4 jets, we use the deep CSV tagger to b-tag the jets. We require deep CSV $>$ 0.6324, found in Figure~\ref{fig:AK4btag}. 

\begin{figure}[h]
\begin{center}
\includegraphics[scale=0.34]{F5/shapebtag1.pdf}
\includegraphics[scale=0.34]{F5/shapebtag2.pdf}
\end{center}
\caption{Deep CSV P(b) + P(bb) tagger discriminant for the highest $p_{T}$ selected AK4 jet (left) and the other selected AK4 jet (right). We require deepCSV $>$ 0.6324.}
\label{fig:AK4btag}
\end{figure} 

\subsection{Variable for tt Reduction}

Lastly, we studied many variables related to unselected AK4 jets present in selected events to determine how to remove as much tt as possible. The variable that provided the most discriminating power was the invariant mass of the two selected AK4 jets combined with the nearest unselected AK4 jet to one of the selected AK4 jets that is not part of the AK8 jet. We call this combined mass the triAK4jet mass. This variable can be seen in Figure~\ref{fig:triak4jetm}, where we place a cut requiring the value of the invariant mass to be larger than 200 GeV. 

\begin{figure}[h]
\begin{center}
\includegraphics[scale=0.5]{F5/shapeinvmsnAK4cl.pdf}
\end{center}
\caption{TriAk4jet mass: the invariant mass of the two selected AK4 jets and the nearest unselected AK4 jet. We require triAK4jet mass $>$ 200 \GeV.}
\label{fig:triak4jetm}
\end{figure} 

%We have then optimized the operating point to use for this search separately for low and high p_{T} jets based on the expected sensitivity. The optimal choice results to be Tight for low and medium p_{T} regime while Loose for high p_{T} jets. An optimization of the double b tagger categories were done using a full event selection and expected limits based on the data. The Alphabet method, with signal region blinded, was used to obtain the background estimate in the signal region. Among the possible combinations taken into account the best sensitivity is achieved when we combine the following two categories:
%\begin{itemize}
%\item TT, both H-jets pass the tight operating point.
%\item LL, both H-jets pass the loose operating point but both fail the tight operating point, i.e. not in the TT category. This also means that if one of the jets passes loose but failes tight, and the other passes the tight operating point, such an event is classified as LL.
%\end{itemize} 

\subsection{Invariant mass definition and ``reduced mass''\label{sss:DijetMassDef}}

The invariant $M_{jjj}$ mass distribution of the AK8 jet and two AK4 jets in the event corresponds to the invariant mass of the resonance searched for. We use instead the reduced mass, defined
% However, it is well known that techniques such as kinematic fit that constraint the mass of each Higgs candidate to $\mH$ improves the resonance resolution and ultimately the sensitivity~\cite{CMS-PAS-B2G-16-008}. This technique was well validated also in the resolved case~\cite{Khachatryan:2015year}. For the boosted case, we considered constraining either the groomed or the ungroomed mass of the Higgs-tagged jets to \mH to improve the resolution of \mjj. A systematic study was performed in Ref.~\cite{CMS-AN-15-265}. In some cases we observed an improvement of the resolution but an over-correction of $\mjj$, \textit{i.e.}, the average position of $\mjj$ shifted well above $\mx$.
%It was found that the variable
\begin{equation}
M_{jjj}^{red} \equiv M_{jjj} - (M_{AK8}-M_{H}) - (M_{1AK4 + 2AK4}-M_{H})
\label{eq:Mjjs}
\end{equation}
since it provides resolution improvement and the mean position of $M_{jjj}^{red}$ remains at $\approx M_{jjj}$. In this equation, $M_{AK8}$ is the softdrop corrected mass of the AK8 jet, $M_{1AK4 + 2AK4}$ is the dijet mass of the two AK4 jets, and $M_{H}$ is 125 GeV. The reduced mass can be found in Figure~\ref{fig:redm}.
We require the reduced mass $>$ 750 GeV because this analysis has little sensitivity below this level.

\begin{figure}[h]
\begin{center}
\includegraphics[scale=0.5]{F5/shaperedmass.pdf}
\end{center}
\caption{Reduced mass of the two selected AK4 jets and the AK8 jet.}
\label{fig:redm}
\end{figure} 

\subsection{Lepton Veto}

A lepton veto is applied. Events are vetoed if they contain one strictly defined or two loosely defined opposite-sign same-flavour isolated electrons or muons with $p_{T}$ in excess of 20 GeV. No veto on the presence of isolated photons is applied. 

\subsection{Trigger Choice \label{ss:trigger}}

As mentioned in the previous chapter, events in a particular dataset are organized by which triggers they pass. In this case, we chose a combination of triggers, and we require that each event pass at least one of these triggers. The triggers chosen place requirements on the scalar sum of jet transverse energy, $H_{T}$, jet $p_{T}$, the jet groomed mass, and b-tagging. The trigger paths used are listed in Table~\ref{tab:trigpaths}. Recall that HLT triggers are sorted by which Level 1 (L1) triggers were passed, which is also listed in the table. The $H_{T}$ triggers are chosen because we expect signal events to have a high $H_{T}$. However, the L1 $H_{T}$ trigger paths were observed to have an inefficiency, so we also used jet $p_{T}$ based triggers to recover these events that did not pass the $H_{T}$ triggers. In the last data run (Run H), the \texttt{HLT\_PFHT800} trigger was prescaled (not all events that passed the trigger were kept, as explained in the previous chapter). Therefore, we added the \texttt{HLT\_PFHT900} trigger to compensate for this.
\begin{table} [htb]
  \begin{center}
    \caption{The HLT paths used and the corresponding L1 seeds.}\label{tab:trigpaths}
    \begin{tabular}{|l|l|}
      \hline
      HLT path & L1 seeds \\
      \hline
       \texttt{PFHT650\_WideJetMJJ900DEtaJJ1p5}              & \texttt{HTT160-255} \\  
       \texttt{AK8PFHT650\_TrimR0p1PT0p03Mass50}             & \texttt{HTT240-320} \\  
       \texttt{AK8PFHT700\_TrimR0p1PT0p03Mass50}             & \texttt{HTT240-320} \\
       \texttt{PFHT800}                                      & \texttt{HTT160-255} \\  
       \texttt{PFHT900}                                      & \texttt{HTT160-255} \\  
       \texttt{AK8PFJet360\_TrimMass30}                      & \texttt{SingleJet180/200} \\  
       \texttt{AK8DiPFJet280\_200\_TrimMass30\_BTagCSV\_p20} & \texttt{SingleJet180/200} \\   
      \hline
    \end{tabular}
  \end{center}
\end{table}
For the L1 triggers, the names of each trigger explain what is required to pass the trigger. "HTT" means that $H_{T}$ in that event is greater than the number that comes after, for example $H_{T} > 160$ GeV. "SingleJet" means that there was at least one jet, with $p_{T}$ greater than the number listed, for example jet 1 has $p_{T} > 180$ GeV. The first HLT trigger requires $H_{T} > 650$ GeV, where two jets have a combined mass $> 900$ GeV and the difference in the pseudorapidity between the two jets is less than 1.5. The second and third HLT trigger require $H_{T} > 650$ GeV and 700 GeV respectively, with a mass jet $>$ 50 GeV. The fourth and fifth trigger require $H_{T} > 800$ GeV and 900 GeV respectively. The sixth trigger requires an AK8 jet with $p_{T} > 360$ GeV, and a mass $> 30$ GeV, while the last trigger requires two AK8 jets with $p_{T} > 280$ GeV and 200 GeV respectively, with mass $> 30$ GeV and b-tag value $>$ 0.20 (using the CSV b-tagging algorithm, the precursor to deepCSV). Most signal events should pass at least one of these triggers.

The trigger requirement is applied to both the data and the MC (signal and tt). In order to compensate for the difference in trigger response between the data and the simulation, trigger efficiency scale factors, defined as the ratio of the trigger efficiency as measured in the data to that in the MC, are applied to the simulated events. Trigger efficiency is the number of events that pass a defined selection and the trigger, divided by the number of events that pass that same defined selection regardless of what triggers they pass. A baseline trigger of \textsc{HLT\_PFJet260}, a prescaled trigger which requires at least one jet with $ p_{T} > 260$ GeV, is used to select events for the measurement of the trigger efficiency. This trigger is prescaled, yet provides enough events for measurement of the efficiencies and the scale factors. Events passing the baseline trigger are further required to pass selection criteria close to the signal selection in the actual analysis:
\begin{itemize}
 \item At least one AK8 jet in the event with $p_{T} > 300\GeV$ and $|\eta| < 2.4$;
 \item At least two AK4 jets with $p_{T} > 30\GeV$ and $|\eta| < 2.4$, with a deep CSV value $>$ 0.6324;
 \item Out of the AK8 and AK4 jets that satisfy the above requirements, one AK8 jet and two AK4 jets are chosen to be the three signal jets. The highest $p_{T}$ AK8 jet with $M_{\rm soft\,drop} > 40$ \GeV that is $\Delta R$ $>$ 0.8 away from two AK4 jets that are within $\Delta R$ $<$ 1.5 of each other is chosen. If there are more than two AK4 jets that satisfy this criteria, the two AK4 jets with the highest deepCSV value are chosen.
 \item The soft drop mass of the AK8 jet is $105 < M_{\rm soft\,drop} < 135\GeV$, with all necessary jet mass corrections applied;
 \item The combined mass of the two AK4 jet is $90 < M_{dijet} < 140\GeV$;
 \item Reduced mass, introduced in Eqn.~\ref{eq:Mjjs}, $>$ 750 GeV, since the analysis has little sensitivity below this value;
 \item $\Delta\eta$ = $|\eta_{AK8} - \eta_{1AK4 + 2AK4}| < 2.0$;
 \item Lepton veto.
\end{itemize}
This selection is used in lieu of the full selection because the full selection can only be applied to data when the analysis has been fully verified, so as to avoid biasing results. In addition to the selection, the trigger efficiency is calculated separately for events with $\Delta\eta < 1.0$ and events with $1.0 \leq \Delta\eta < 2.0$. Then the trigger efficiency for data and MC is defined as 
\begin{equation}
\frac{\text{N. events passing selection + \textsc{HLT\_PFJet260} + at least one trigger listed}}{\text{N. events passing selection + \textsc{HLT\_PFJet260} }}
\end{equation}
Trigger efficiency is measured as a function of the ``reduced mass'' introduced in Eqn.~\ref{eq:Mjjs} of Section~\ref{sss:DijetMassDef}. The efficiency is shown in Figure~\ref{fig:trigeEffvsMjj_JetHT_DetaBins}. The trigger efficiency at low mass in the high $\Delta\eta$ region is higher than expected due to the inefficiency of \textsc{HLT\_PFJet260} at low mass. 

\begin{figure}[h]
  \begin{center}  
    \includegraphics[width=0.45\textwidth]{F5/trigefDeta0v2.pdf} 
    \includegraphics[width=0.45\textwidth]{F5/trigefDeta1v2.pdf} 
 %   \includegraphics[width=0.45\textwidth]{F5/JetHT_QCD_mjjred_py0_trigEff.pdf} 

 %   \includegraphics[width=0.45\textwidth]{F5/JetHT_QCD_mjjred_py1_trigEff.pdf} 
 %   \includegraphics[width=0.45\textwidth]{F5/JetHT_QCD_mjjred_py2_trigEff.pdf} 
  \end{center}
  \caption{The trigger efficiency, as a function of reduced mass. Efficiency is defined for different $\Delta\eta$ regions: 0.0--1.0 (left), 1.0--2.0 (right).}%, 0.434--0.868 (lower left), and 0.868--1.3 (lower right). 
 %The 23Sep2016 re-reco for Runs2016B-G and PromptReco for Run2016H are used, totalling to an integrated luminosity of \intLumi.}
  \label{fig:trigeEffvsMjj_JetHT_DetaBins}
\end{figure}

%The combined set of triggers reaches full efficiciency for $\mjjs > 1000\GeV$ over all the $\Delta\eta$ ranges. 
For $M_{jjj}^{red} < 1000$ GeV, the trigger efficiencies are higher for smaller $\Delta\eta$, where most of the signal lie, and are smaller at larger values of $\Delta\eta_{jj}$. Furthermore, the data/MC scale factor too varies depending on $\Delta\eta_{jj}$. Since we begin the search from $M_{jjj}^{red}$ well below 1000\GeV, one needs to be especially careful of the modelling of the trigger efficiency turn-on curves in the data and the simulations. Since the baseline trigger \textsc{HLT\_PFJet260} too has some inefficiency for low $M_{jjj}^{red}$, we measure it in QCD MC and assign an uncertainty to the trigger efficiency scale factor based on this. This efficiency is defined as
\begin{equation}
\frac{\text{N. events passing selection + \textsc{HLT\_PFJet260}}}{\text{N. events passing selection}}
\end{equation}
\begin{figure}[h]
  \begin{center}  
    \includegraphics[width=0.45\textwidth]{F5/QCDDeta0v2.pdf} 
    \includegraphics[width=0.45\textwidth]{F5/QCDDeta1v2.pdf} 
  \end{center}
  \caption{The trigger efficiency in QCD MC for the baseline trigger \textsc{HLT\_PFJet260}, for different $\Delta\eta$ regions: 0.0-1.0 (left) and 1.0-2.0 (right).%, for different $\Delta\eta_{jj}$ regions: 0.0--1.3, 0.0--0.434, 0.434--0.868, and 0.868--1.3. 
The percentage difference between one and these turn-on curves are taken as an uncertainty on the trigger efficiency scale factor.}
  \label{fig:trigeEffvsMjj_QCDHT_HLTPFJet260_DEtabins}
\end{figure}
Figure~\ref{fig:trigeEffvsMjj_QCDHT_HLTPFJet260_DEtabins} shows the \textsc{HLT\_PFJet260} trigger turn-on curves for different $\Delta\eta$ regions in MC, %for different $\Delta\eta_{jj}$ in MC, 
The difference between unity and the trigger efficiency is propagated to the scale factor as a systematic uncertainty. In addition to this systematic uncertainty, we assign a 0.5\% uncertainty based on the use of the last trigger listed in Table~\ref{tab:trigpaths}, since it has a b-tagging requirement. The trigger efficiency difference between data and MC is calculated comparing QCD and data, but QCD has a lower percentage of b-quarks than signal does, and this trigger efficiency scale factor gets applied to signal, so we had to correct for this difference in composition.
The trigger efficiency scale factor is calculated as
\begin{equation}
\frac{\text{efficiency in data}}{\text{efficiency in MC}}
\end{equation}
The scale factor and errors are shown in Figure~\ref{fig:trigSF}. 
\begin{figure}[h]
  \begin{center}  
    \includegraphics[width=0.45\textwidth]{F5/SFdEta0v2.pdf} 
    \includegraphics[width=0.45\textwidth]{F5/SFdEta1v2.pdf} 
%    \includegraphics[width=0.33\textwidth]{F5/c_gsftrigger_0p0To0p434} 
%    \includegraphics[width=0.33\textwidth]{F5/c_gsftrigger_0p868To1p3} 
  \end{center}
  \caption{The trigger efficiency scale factors, as a function of $M_{jjj}^{red}$. The error bars are the combined statistical and systematic errors.}
  \label{fig:trigSF}
  \end{figure} 

%\subsection{Cutflow}

%The cut flow is given in Table~\ref{tab:sigeff} for bulk gravitons and radions. %and in Table~\ref{tab:RadionCutFlowEff} for radions. 
%The main difference in the efficiencies of the spin-2 and the spin-0 models stems from the different $\Delta\eta_{jj}$ distributions, that for the bulk graviton being more central. Thus the bulk gravitions have a higher efficiency than the radions.

%\begin{table}[h]
%\caption{Fraction of signal events left after cuts for resonant bulk graviton and radion. Preselection is the second column (excluding the trigger), trigger is the third column, AK4 Deep CSV Medium tag is the fourth, AK4 dijetmass requirement is the fifth, AK8 puppi $\tau_{21}$ is the sixth, triAk4jet mass is the seventh, AK8 softdrop corrected mass is the eigth, AK8 double-b tag is the ninth, reduced mass is the tenth, $\Delta\eta$ is the eleventh. Full selection efficiency is represented by the eleventh column, selection efficiency after full selection and rejecting events that pass the boosted selection is the twelfth, and selection efficiency after full selection and rejecting events that pass the resolved selection events is the last.}\label{tab:sigeff}
%\scalebox{0.55}{
%\begin{tabular}{|l|c|c|c|c|c|c|c|c|c|c|c|c|c|}
%\hline
%Mass & Presel & Trig & AK4btag & AK4dijetM & $\tau_{21}$ & TriAK4JetM & AK8M & AK8btag & Reduced M & $\Delta\eta$& Boosted & Resolved\\ \hline
%500 & 0.068 & 0.046 & 0.029 & 0.013 & 0.01 & 0.008 & 0.003 & 0.001 & 0.001 & 0.001\\
%550 & 0.112 & 0.064 & 0.04 & 0.021 & 0.016 & 0.014 & 0.005 & 0.003 & 0.003 & 0.001\\
%600 & 0.188 & 0.093 & 0.059 & 0.032 & 0.026 & 0.022 & 0.01 & 0.005 & 0.005 & 0.002\\
%650 & 0.292 & 0.132 & 0.082 & 0.048 & 0.04 & 0.034 & 0.017 & 0.009 & 0.009 & 0.004\\
%750 & 0.466 & 0.347 & 0.218 & 0.14 & 0.123 & 0.108 & 0.063 & 0.039 & 0.029 & 0.013\\
%800 & 0.516 & 0.446 & 0.277 & 0.179 & 0.159 & 0.141 & 0.085 & 0.052 & 0.03 & 0.014\\
%900 & 0.578 & 0.554 & 0.334 & 0.218 & 0.193 & 0.174 & 0.108 & 0.065 & 0.031 & 0.016\\
%1000 & 0.614 & 0.603 & 0.357 & 0.235 & 0.209 & 0.192 & 0.122 & 0.072 & 0.032 & 0.017\\
%1200 & 0.624 & 0.622 & 0.354 & 0.237 & 0.213 & 0.199 & 0.132 & 0.078 & 0.036 & 0.023\\
%1600 & 0.39 & 0.39 & 0.181 & 0.106 & 0.094 & 0.089 & 0.059 & 0.033 & 0.023 & 0.021\\
%2000 & 0.252 & 0.251 & 0.09 & 0.042 & 0.037 & 0.036 & 0.024 & 0.014 & 0.012 & 0.012\\
%BG 750 & 0.202 & 0.323 & 0.204 & 0.168 & 0.149 & 0.12 & 0.074 & 0.046 & 0.043 & 0.043 & 0.033 & 0.022\\
%BG 800 & 0.236 & 0.425 & 0.265 & 0.22 & 0.196 & 0.161 & 0.101 & 0.062 & 0.061 & 0.061 & 0.039 &0.032\\
%BG 900 & 0.279 & 0.541 & 0.326 & 0.276 & 0.245 & 0.208 & 0.133 & 0.08 & 0.08 & 0.08 & 0.043 &0.044\\
%BG 1000 & 0.302 & 0.596 & 0.353 & 0.3 & 0.267 & 0.232 & 0.149 & 0.088 & 0.088 & 0.087 & 0.046 &0.051\\
%BG 1200 & 0.311 & 0.619 & 0.352 & 0.298 & 0.267 & 0.239 & 0.158 & 0.093 & 0.093 & 0.092 & 0.047 &0.064\\
%BG 1600 & 0.195 & 0.39 & 0.182 & 0.135 & 0.121 & 0.112 & 0.074 & 0.041 & 0.041 & 0.039 & 0.028 &0.037\\
%BG 2000 & 0.127 & 0.254 & 0.091 & 0.055 & 0.049 & 0.046 & 0.031 & 0.017 & 0.017 & 0.015 & 0.014 &0.015\\ \hline
%Rad 750 & 0.14 & 0.213 & 0.133 & 0.108 & 0.096 & 0.078 & 0.047 & 0.03 & 0.027 & 0.027 & 0.021 &0.014\\
%Rad 800 & 0.169 & 0.291 & 0.181 & 0.15 & 0.134 & 0.11 & 0.068 & 0.041 & 0.041 & 0.04 & 0.026 &0.022\\
%Rad 1000 & 0.237 & 0.457 & 0.273 & 0.231 & 0.206 & 0.178 & 0.112 & 0.067 & 0.067 & 0.067 & 0.038 &0.039\\
%Rad 1200 & 0.263 & 0.516 & 0.291 & 0.243 & 0.215 & 0.189 & 0.12 & 0.069 & 0.069 & 0.066 & 0.037 & 0.046\\
%Rad 1400 & 0.241 & 0.476 & 0.257 & 0.211 & 0.189 & 0.168 & 0.108 & 0.063 & 0.063 & 0.056 & 0.037 & 0.046 \\
%Rad 1600 & 0.205 & 0.407 & 0.2 & 0.157 & 0.14 & 0.127 & 0.079 & 0.044 & 0.044 & 0.035 & 0.027 &0.031\\
%\hline
%\end{tabular}
%}
%\end{table}

\subsection{Summary of the Semi-Resolved Section}

Taking into account all of the above information, we can summarize the selection as follows:

\begin{itemize}
\item Passes triggers described in Table~\ref{tab:trigpaths};
\item One AK8 jets, $p_{T} > 300$ GeV, $|\eta| < 2.4$, tight jet ID;
\item Two AK4 jets, $p_{T} > 30$ GeV, $|\eta| < 2.4$;
\item Out of the AK8 and AK4 jets that satisfy the above requirements, one AK8 jet and two AK4 jets are chosen to be the three signal jets. The highest $p_{T}$ AK8 jet with $M_{\rm soft\,drop} > 40$ \GeV that is $\Delta R$ $>$ 0.8 away from two AK4 jets that are within $\Delta R$ $<$ 1.5 of each other is chosen. If there are more than two AK4 jets that satisfy this criteria, the two AK4 jets with the highest deepCSV value are chosen.
\item $|\Delta\eta| < 2.0$ between the AK8 jet and the combined AK4 jets;
\item $M_{jjj}^{red}$ $>$ 750 GeV;
\item triAK4jet mass $>$ 200 GeV;
\item Lepton veto;
\item The soft drop mass of the AK8 jet is $105 < M_{\rm soft\,drop} < 135\GeV$, with all necessary jet mass corrections applied;
\item 90 $<$ Dijet mass of the two AK4 jets $<$ 140 GeV
\item double b-tagger $>$ 0.8
\item deep CSV $>$ 0.6324
\item $\tau_{21}$ $<$ 0.55 for the AK8 jet;
\end{itemize}

\subsection{Selection of Other Analyses\label{ss:boostres}}

\subsubsection{Boosted Analysis\label{sss:boost}}

The boosted analysis selection is documented in Reference~\cite{Sirunyan:2017isc}.

The selection is identical to the boosted side of the semi-resolved analysis by design; this analysis came first and established best practices for this channel, and the semi-resolved analysis was modeled after the work done in this analysis. The boosted analysis requires:
\begin{itemize}
\item Passes triggers described in Table~\ref{tab:trigpaths};
\item Two AK8 jets, $p_{T} > 300$ GeV, $|\eta| < 2.4$, tight jet ID;
\item $|\Delta\eta| < 1.3$ between two AK8 jets;
\item $M_{jj}^{red}$ $>$ 750 GeV;
\item Lepton veto;
\item The soft drop mass of both AK8 jets is $105 < M_{\rm soft\,drop} < 135\GeV$, with all necessary jet mass corrections applied;
\item Falls into one of the following categories: both AK8 jets have double-b $>$ 0.8, one AK8 jet has double-b $>$ 0.8 and the other has 0.3 $<$ double-b $<$ 0.8, both AK8 jets have 0.3 $<$ double-b $<$ 0.8.
\end{itemize}

It should be noted, however, that this is only the events in the signal region. Because the boosted analysis uses the same background estimate method as this analysis (described in later chapters), we must also account for any event that was used to estimate the background. This way, we prevent any overlap of events, whether in the data signal region or in a region of data used to estimate the QCD background contribution. This set of events is as follows:
\begin{itemize}
\item Passes triggers described in Table~\ref{tab:trigpaths};
\item Two AK8 jets, $p_{T} > 300$ GeV, $|\eta| < 2.4$, tight ID;
\item $|\Delta\eta| < 1.3$ between two AK8 jets;
\item $\tau_{21}$ $<$ 0.55 for both AK8 jets;
\item $M_{jj}$ $>$ 750 GeV;
\item The soft drop mass of \textbf{only the subleading AK8 jet} is required to be between $105 < M_{\rm soft\,drop} < 135\GeV$, with all necessary jet mass corrections applied;
\item The double-b of \textbf{only the subleading AK8 jet} is required to satisfy double-b $>$ 0.3.
\end{itemize}
We note that since the entire mass spectrum and double-b spectrum of the leading AK8 jet is used either in the signal region or as part of the background estimate, we cannot place any restrictions on these values. Then this selection defined directly above is the selection rejected when limits are calculated to be combined with the fully boosted 
analysis. 

\subsubsection{Resolved Analysis\label{sss:resolve}}
The fully resolved non-resonant and resonant analysis have different selections. However, in this thesis, we only talk about combining with the fully resolved non-resonant analysis, so this is the only relevant selection.

Selection is as follows:
\begin{itemize}
\item Passes the triggers HLT\_QuadJet45\_TripleBTagCSV\_p087 \\and/or HLT\_DoubleJet90\_Double30\_TripleBTagCSV\_p087;
\item Four AK4 jets, $p_{T} > 30$ GeV, $|\eta| < 2.4$, and CMVA $>$ 0.4432 (a b-tagger similar to deepCSV)
\end{itemize}

This selection is relatively loose because they optimize events differently, using a boosted decision tree (type of machine learning) to decide which events to keep.

