\chapter{Looking for New Physics}\label{Sec:BSM}

In the previous chapters we learned about the Standard Model, interesting new physics that we can test for at the LHC related to the Higgs boson, and the design of the LHC and CMS. Armed with all of this information, we can now understand how to go about finding this new physics (or proving that certain models are wrong).
\vspace{5mm}

\section{Signal Signature}

We are looking for events that have two Higgs bosons. Inside the detector, the Higgs boson immediately decays, and we observe its decay products rather than the Higgs itself. The most common decay mode for a Higgs boson is into two b quarks, so we choose this for our search. Since there are two Higgs bosons, we are looking for events with four b-quarks. B-quarks show up in the CMS detector as jets. However, the number of jets we expect to see is dependent upon how much momentum each Higgs boson has. The more momentum a particle has, the more collimated its decay products tend to be. In Figure~\ref{Fig:lorentzboost}, we see that if a particle is produced with no momentum (left), its decay products will be back-to-back in the detector. However, as a particle is produced with higher (middle) and even higher (right) momentum, this Lorentz boost causes the decay products to become collimated. 
\begin{figure}[h!]
    \centering
        \includegraphics[width=0.8\textwidth]{F4/boost.png}
        \caption{Drawing of decay products with varying degrees of Lorentz boost.}
        \label{Fig:lorentzboost}
\end{figure}

We know that the minimum energy E required to create two Higgs bosons is 2$\text{M}_{H}$ = 250 GeV. If the energy E $\sim$ 250 GeV or slightly greater, then we expect each Higgs boson to be produced with little momentum. This means that the decay products of each Higgs will not be collimated. Therefore, we would expect to see an event with four distinct AK4 (small) jets, most likely paired off but not too close to any other jet. We would expect the event signature to look something like that of Figure~\ref{Fig:threecases1}. This analysis, called the resolved case, is important, but is being performed by other CMS collaborators, so we will only refer to this analysis when speaking about combining results across the different scenarios.
\begin{figure}[h!]
    \centering
       \includegraphics[width=0.7\textwidth]{F4/resolved.png}
        \caption{Signature for resolved two Higgs to four b-quarks.}
        \label{Fig:threecases1}
\end{figure}

On the other hand, it is also possible that the energy of the collision E $>>$ 2$\text{M}_{H}$. Both Higgs bosons have considerable momentum in this case, so the decay products will be collimated. Rather than two distinct AK4 jets per Higgs, we expect these AK4 jets to merge into one AK8 (large) jet. This means that these events would have two large jets, each representing one Higgs, similar to Figure~\ref{Fig:threecases2}. These events are called boosted events.
\begin{figure}[h!]
    \centering
        \includegraphics[width=0.7\textwidth]{F4/merged.png}
        \caption{Signature for boosted two Higgs to four b-quarks.}
        \label{Fig:threecases2}
\end{figure}

Lastly, given these two extremes, we expect some events to have an energy E that is in the middle of these two cases, such that one Higgs boson is produced with enough momentum to collimate two AK4 jets into one big AK8 while the other Higgs boson does not have enough momentum to have collimated decay products and is reconstructed as two small AK4 jets (Figure~\ref{Fig:threecases3}). We then expect one AK8 jet which is far away from two AK4 jets which are close to each other, but not so close as to have merged into one AK8 jet. We call these events semi-resolved since one half of the event looks like the boosted case and the other half looks like the resolved case. 
\begin{figure}[h!]
    \centering
        \includegraphics[width=0.7\textwidth]{F4/semiresolved.png}
        \caption{Signature for semi-resolved two Higgs to four b-quarks.}
        \label{Fig:threecases3}
\end{figure}
In this thesis, we will focus on the semi-resolved case and its combination with the boosted case.
\vspace{5mm}

\section{Distinguishing Signal from Background}

There will exist background events, which have properties similar to signal, but are known SM contributions to the total number of events we observe. In order to observe signal events in the midst of this background or rule out the possibility of this signal existing, we must optimize the ratio of signal events to background events. The main background for any hadronic analysis is multi-jet QCD events caused by random hadronic activity. This is quite common at the LHC, and comprises the majority of events observed. Signal, or two Higgs decaying to four b quarks (HH$\rightarrow$bbbb), on the other hand, is quite rare. In this section, we will cover several different tools that provide a good way to reduce background events while retaining as many signal events as possible.


\subsection{The Soft-Drop Mass Algorithm}
One of the best discriminating variables is the invariant mass of a given jet or combination of jets. We expect that any AK8 jet representing a Higgs boson should have a mass close to the measured Higgs mass of 125 GeV. The mass of a particular jet is obtained by summing up the momentum four-vectors of all jet constituents into a combined object and then calculating the invariant mass of this object. While we expect signal jets to have a mass near 125 GeV, background processes will have a much wider range of mass values, so this variable has good signal-background discrimination.

There are several algorithms that remove constituents resulting from pileup from jets, making it less likely that background QCD processes would have a mass similar to that of the Higgs. For this analysis we use the Soft-Drop Mass Algorithm \cite{Larkoski:2014wba}. This algorithm begins by undoing the most recent clustering step in the jet algorithm. Then there exists two pseudo-jets: the main one (jet 1) and the last constituent to be added to the main one (jet 2). These two jets are evaluated based on the following equation
\begin{equation}
   \frac{\textrm{min}(p_{T,1},p_{T,2})}{p_{T,1} + p_{T,2}} > z
\end{equation}
The value of $z$ determines how stringent the grooming algorithm is; in our case, we use $z$=0.1. If the above condition is not met, that is to say the lowest $p_{T}$ of the two pseudo-jets is not at least 10\% of the total $p_{T}$ of the two consitutents, then the lowest $p_{T}$ constituent is discarded. This process is repeated, unclustering the jet step by step, until the above condition is met. At that time, the mass of the remaining constituents is defined as the soft drop mass. This is meant to remove, or "drop", low momentum, or "soft", constituents with the assumption that these are more likely to be a result of QCD background jets rather than actual decay products of a signal jet. The difference between ungroomed mass (left) and soft drop mass (right) can be seen comparing a W jet (in this case, signal) and a gluon jet (background), in Figure~\ref{Fig:Tag:SD}~\cite{Adams:2015hiv}. Soft-drop mass provides a more narrowly peaked distribution for signal, while shifting the background peak lower, improving the discrimination power of mass.
\begin{figure}[h!]
    \centering
        \includegraphics[width=0.4\textwidth]{F4/ungroomed.PNG}
        \includegraphics[width=0.4\textwidth]{F4/sd.PNG}
        \caption{Softdrop mass (right) does a better job of discriminating between QCD (blue) and signal (W boson, in red), than ungroomed mass does (left).}
        \label{Fig:Tag:SD}
\end{figure}

\vspace{5mm}
\subsection{The N-subjetiness Algorithm}

Signal jets tend to have a certain amount of substructure, while background jets tend to be more chaotic. For a Higgs jet that decays to two b-quarks, we expect that if the Higgs jet is contained within one AK8 jet, there would be two subjets within that AK8 jet, each representing one of the b-quarks.  The N-subjetiness Algorithm~\cite{Thaler:2010tr,Thaler:2011gf} defines the likelihood of a given number of subjets in a jet, allowing for an easier discrimination between jets with substructure and jets without. The algorithm compares different jet constituents to subjet axes (direction in which a particular subjet points) to determine how likely it is that a jet has a certain number of subjets. This is done by defining $\tau_N$, 
\begin{equation}
\tau_N = \frac{1}{d_0}\sum_i p_{T_i}\times min(\Delta R_{1,i},\Delta R_{2,i},...,\Delta R_{N,i})
\end{equation}
where N is the number of subjets, $\Delta R$ measures the distance between a given subjet axis and a consituent $i$, and $d_0 = \sum p_{T_i} R_0$ where $p_{T_i}$ is the momentum of a constituent and $R_0$ is the radius of the jet (0.8 for AK8 jets). This then measures the sum of each constituent's $p_{T}$ multiplied by the distance to the closest subjet axis, divided by the sum of each constituent's $p_{T}$ multiplied by the jet radius. 

For example, a jet with two subjets would have a low $\tau_{2}$ value, since the constituents would have low $\Delta R_{i,j}$ because they would be near one of the two $\tau$ axes. However, this jet would have a high $\tau_{1}$ value since the constituents may not necessarily be near the one $\tau$ axis. Since we are looking for jets with two subjets, and attempting to reject jets with no substructure (ie one subjet, which is just the jet itself), we compare $\tau_2/\tau_1$. We expect this to be close to 0 for signal, since $\tau_{2}$ should be lower than $\tau_{1}$. However, for QCD, we would expect that $\tau_{2}$ should be higher, since it tends to not have any substructure. The closer to 0 $\tau_2/\tau_1$ is, the more likely the jet is to have two subjets and be signal-like, whereas the closer to 1, the more likely the jet is to have no substructure, or one subjet, and be background-like.

This variable can be seen in Figure~\ref{Fig:tau21}~\cite{Khanpour:2014xla}, where W is also expected to have two subjets when it decays hadronically, but QCD is not expected to have much substructure.
\begin{figure}[h!]
    \centering
        \includegraphics[width=0.66\textwidth]{F4/tau21.png}
        \caption{N-subjetiness variable $\tau_2/\tau_21$ for W jets and background jets.}
        \label{Fig:tau21}
\end{figure}
\vspace{5mm}

\subsection{B-Tagging}
We expect to see four b-quarks in each event for the signal HH$\rightarrow$bbbb. For the semi-resolved case, this means that the resolved Higgs should have two AK4 jets which are each from a b-quark, and the boosted Higgs should have one AK8 jet with two b-quark subjets inside. Because b-quarks leave a signature in the detector that is unique from any other hadronic activity, we are able to distinguish b-jets from jets originating from lighter quarks. Bare quarks immediately form hadrons. The lightest quarks hadronize immediately, so their jets are formed at the interaction point, or primary vertex (Figure~\ref{Fig:bjet}, left). However, b hadrons have a longer lifetime, so they travel at the order of millimeters in the detector before decaying (Figure~\ref{Fig:bjet}, right). This means that there exists both a primary and secondary vertex which can be seen in the detector. 
\begin{figure}[h!]
    \centering
        \includegraphics[width=0.7\textwidth]{F4/bquark.png}
        \caption{Light quark (u, d, c, s) in a detector (left) vs. b-quark in a detector (right).}
        \label{Fig:bjet}
\end{figure}
Conversely, a top quark has such a short lifetime that it does not have any time to hadronize and immediately decays. Thus, a b-quark has a unique signature in comparison with the other five quarks and this signature can be used to identify jets coming from b-quarks. We use two different algorithms for identifying b-quarks in this analysis: the Deep CSV algorithm identifies AK4 b jets and the Double-b Algorithm identifies AK8 jets with two b-quarks inside. A full description of b-tagging algorithms used by CMS can be found in Reference~\cite{Sirunyan:2017ezt}.

\subsubsection{Deep CSV Algorithm}

In order to identify AK4 jets as b-jets, we use the deep CSV algorithm, a multivariate measurement of jets with a secondary vertex, which takes into account information from the displaced tracks and from the secondary vertices associated with the jet. The algorithm is trained to recognize b-jets and non b-jets depending on these variables, using a deep neural network to "learn" the difference between signal and background so as to assign a certain likelihood of a jet containing a b-quark based off of these different variables (0 is unlikely, 1 is very likely). The discriminator can be seen in Figure~\ref{Fig:dcsv}, and is a combination of the probability of a jet containing one b hadron and the probability of a jet containing two b hadrons. Overall, this algorithm outperforms the other b-tagging algorithms used on CMS data.
\begin{figure}[h!]
    \centering
        \includegraphics[width=\textwidth]{F4/dcsv.png}
        \caption{Deep CSV discriminator for b jets (red), c jets (green), and light jets (blue).}
        \label{Fig:dcsv}
\end{figure}

\subsubsection{Double-b Algorithm}

The double-b algorithm is designed to identify AK8 jets with two b-quarks inside of the jet. It was specifically designed to identify H$\rightarrow$bb jets. Just like the deep CSV algorithm, the double-b algorithm uses a multivariate approach to identify jets with two b-quarks, taking into account variables related to the $\tau$ axes calculated by the N-subjettiness algorithm as well as variables related to the displaced tracks and secondary vertices. In particular, this algorithm is designed to be $p_{T}$ and mass independent, which allows for a wide range of AK8 jets to be properly tagged. The algorithm is trained to recognize jets with two b-quarks and jets without two b-quarks depending on the aforementioned variables, using a boosted decision tree to "learn" the difference between signal and background. The algorithm assigns a likelihood of a jet containing two b-quarks, ranging from -1 (unlikely) to 1 (very likely). This can be seen in Figure~\ref{Fig:doubleb}. The double-b algorithm outperforms any other tagger used by CMS to identify jets with two b-quarks when the background mostly consists of QCD or jets with two b-quarks coming from gluon fusion, as well as when jets have high $p_{T}$. Since this is our case, this algorithm was chosen to identify the AK8 jet with two b-quarks that are arising from a Higgs decay.
\begin{figure}[h!]
    \centering
        \includegraphics[width=\textwidth]{F4/doubleb.png}
        \caption{Double b-tagger for signal (red) and various QCD processes (varying shades of blue).}
        \label{Fig:doubleb}
\end{figure}

