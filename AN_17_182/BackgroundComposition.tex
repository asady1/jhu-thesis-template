\section{DataMC Comparison\label{sec:BkgComp}}

%In order to study the background composition of the events passing our selection criteria we have used two different control samples. We have built them by inverting either the $\tau_{21}$ or the double-b tag cuts for exactly one of the two selected jets.

%\subsection{\texorpdfstring{$\tau_{21}$}{tau21} inverted control region\label{ss:t21control}}

The same preselection applied to the shape plots is applied to the following plots as described in Sec.~\ref{ss:JetSel}, comparing data, QCD, and ttbar. However, the following cuts are applied to create a control region that is more similar to the signal region while leaving the signal region blind:
\begin{itemize}
\item 
\item Rejecting events in the softdrop mass window $105 < M_{\rm soft\,drop} < 135 \GeV$;
\item Enforcing the dijet mass window $105 < M_{dijet} < 135 \GeV$;
\item Enforcing the deep CSV medium point, requiring deep CSV $> 0.6324$ for each AK4 jet.
 \item Reduced mass, introduced in Eqn.~\ref{eq:Mjjs}, $>$ 700 GeV, since the analysis has little sensitivity below this value;
 \item $\Delta\eta$ = $|AK8\eta - (AK4jet1 + AK4jet2)\eta| < 2.0$;
\end{itemize} 
Data MC agreement is good across variables, as can be seen in Figs.~\ref{fig:dMCAK8pteta}-~\ref{fig:dMCredm}. 

\begin{figure}[thb!]
\begin{center}
\includegraphics[scale=0.34]{Figures/ratMCptFJ.pdf}
\includegraphics[scale=0.34]{Figures/ratMCetaFJ.pdf}\\
\end{center}
\caption{Left. The $\pt$ of the AK8 jet. Right. The $\eta$ of the AK8 jet.}
\label{fig:dMCAK8pteta}
\end{figure} 

\begin{figure}[thb!]
\begin{center}
\includegraphics[scale=0.34]{Figures/ratMCptJ1.pdf}
\includegraphics[scale=0.34]{Figures/ratMCetaJ1.pdf}\\
\end{center}
\caption{Left. The $\pt$ of the highest $\pt$ selected AK4 jet. Right. The $\eta$ of the highest $\pt$ selected AK4 jet.}
\label{fig:dMCAK41pteta}
\end{figure} 

\begin{figure}[thb!]
\begin{center}
\includegraphics[scale=0.34]{Figures/ratMCptJ2.pdf}
\includegraphics[scale=0.34]{Figures/ratMCetaJ2.pdf}\\
\end{center}
\caption{Left. The $\pt$ of the other selected AK4 jet. Right. The $\eta$ of the other selected AK4 jet.}
\label{fig:dMCAK42pteta}
\end{figure} 

\begin{figure}[thb!]
\begin{center}
\includegraphics[scale=0.34]{Figures/ratMCdeta.pdf}
\end{center}
\caption{The $\Delta\eta$ between the AK8 jet and the combined 4 vector of the two AK4 jets.}
\label{fig:dMCdeta}
\end{figure} 

\begin{figure}[thb!]
\begin{center}
\includegraphics[scale=0.5]{Figures/ratMCjetmass.pdf}
\end{center}
\caption{The soft-drop mass of the AK8 jet.}
\label{fig:dMCAK8mass}
\end{figure} 

\begin{figure}[thb!]
\begin{center}
\includegraphics[scale=0.5]{Figures/ratMCdijetmass.pdf}
\end{center}
\caption{The dijet mass of the AK4 jet.}
\label{fig:dMCAK4dijetmass}
\end{figure}

\begin{figure}[th!b]
\begin{center}
\includegraphics[scale=0.5]{Figures/ratMCptau21.pdf}
\end{center}
\caption{$\nsub$ distribution for the AK8 jet.\label{fig:dMCtau21}}
\end{figure}

\begin{figure}[h]
\begin{center}
\includegraphics[scale=0.5]{Figures/ratMCdoubleb.pdf}
\end{center}
\caption{Double-b tagger discriminant for the AK8 jet.}
\label{fig:dMCdoubleb}
\end{figure} 

\begin{figure}[h]
\begin{center}
\includegraphics[scale=0.34]{Figures/ratMCbtag1.pdf}
\includegraphics[scale=0.34]{Figures/ratMCbtag2.pdf}
\end{center}
\caption{Deep CSV P(b) + P(bb) tagger discriminant for the highest $\pt$ selected AK4 jet (left) and the other selected AK4 jet (right).}
\label{fig:dMCAK4btag}
\end{figure} 

\begin{figure}[h]
\begin{center}
\includegraphics[scale=0.5]{Figures/ratMCinvmsnAK4cl.pdf}
\end{center}
\caption{TriAK4jet mass: the invariant mass of the two selected AK4 jets and the nearest unselected AK4 jet.}
\label{fig:dMCtriak4jetm}
\end{figure} 

\begin{figure}[h]
\begin{center}
\includegraphics[scale=0.5]{Figures/ratMCredmass.pdf}
\end{center}
\caption{Reduced mass of the two selected AK4 jets and the AK8 jet.}
\label{fig:dMCredm}
\end{figure} 


%The $\tau_{21}$ inverted control sample has been selected by applying all of the event selection criteria but the $\tau_{21}$ requirement; instead we require either only leading jet or only subleading jet passes the $\tau_{21}$ requirement. 
%In this comparison no SF is applied and simulation yield is scaled to match the one in data (data/MC is $\approx 1.26$) in Figures \ref{fig:dMCjet1_tau_inverted} and \ref{fig:jet2_tau_inverted}.
%The major background from QCD multi-jet production is simulated and compared with the data. We have split this contribution by flavor content.

%\begin{figure}[h]
%\centering
%\includegraphics[width=0.45\textwidth]{Figures/anti_tau21_rereco_doublebtagv4/puppiSDMassThea_j0.pdf}
%\includegraphics[width=0.45\textwidth]{Figures/anti_tau21_rereco_doublebtagv4/pt_j0.pdf}
%\includegraphics[width=0.45\textwidth]{Figures/anti_tau21_rereco_doublebtagv4/eta_j0.pdf}
%\includegraphics[width=0.45\textwidth]{Figures/anti_tau21_rereco_doublebtagv4/doubleSV_j0.pdf}
%\caption{ Comparison plots of data/simulation for kinematic observables and double-b tag value for the leading jet in the $\tau_{21}$ inverted cont%rol sample. From left to right: Thea-corrected softdrop mass, $\pt$, $\eta$ and double-b tag discriminator.}
%\label{fig:jet1_tau_inverted}
%\end{figure}


