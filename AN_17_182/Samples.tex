\section{Data and simulated samples\label{sec:Samples}}

\subsection{Data samples\label{ss:DataSamples}}

The analysis is performed using $\Pp\Pp$ interactions collected with the CMS detector at $\sqrt{s}=13~\TeV$. The data correspond an integrated luminosity of \intLumi, as measured using the golden JSON file \href{https://cms-service-dqm.web.cern.ch/cms-service-dqm/CAF/certification/Collisions16/13TeV/ReReco/Final/Cert_271036-284044_13TeV_23Sep2016ReReco_Collisions16_JSON.txt}{Cert\_271036-284044\_13TeV\_23Sep2016ReReco\_Collisions16\_JSON.txt}. The data samples are summarized in Table~\ref{tab:data}.

\begin{table}[htbH]\footnotesize
  \begin{center}
    \topcaption{List of primary datasets for the \Pp\Pp\,collisions at $\sqrt{s} = 13~\TeV$ and their corresponding integrated luminosities. The golden JSON file \href{https://cms-service-dqm.web.cern.ch/cms-service-dqm/CAF/certification/Collisions16/13TeV/ReReco/Final/Cert_271036-284044_13TeV_23Sep2016ReReco_Collisions16_JSON.txt}{Cert\_271036-284044\_13TeV\_23Sep2016ReReco\_Collisions16\_JSON.txt} was used.}
    \begin{tabular}{l|c|c}
      \hline
      \hline
      Dataset & Processing & Int. lumi. (fb$^{-1}$) \\
      \hline
      JetHT/Run2016B   & 03Feb2017 & 5.9  \\
      JetHT/Run2016C   & 03Feb2017 & 2.6  \\
      JetHT/Run2016D   & 03Feb2017 & 4.4  \\
      JetHT/Run2016E   & 03Feb2017 & 4.1  \\
      JetHT/Run2016F   & 03Feb2017 & 3.2  \\
      JetHT/Run2016G   & 03Feb2017 & 7.7  \\ 
      JetHT/Run2016H   & 03Feb2017 & 8.9 \\ 
      \hline
      Total & & \intLumi \\ 
      \hline
      \hline  
    \end{tabular}  
    \label{tab:data}
  \end{center}
\end{table}

\subsection{MC simulation\label{ss:MCSimulation}}

The MC samples used for this analysis includes spin-0 bulk graviton and spin-2 radion resonances,
%and non resonant bulk graviton, 
as given in Table~\ref{tab:signal_MC}, as well as multijet and \ttbar MC, as given in Table~\ref{tab:bkg_MC}. The Moriond2017 pileup scenario was used to approximate the number of inelastic collisions per bunch crossing at LHC 13~\TeV data-taking using an inelastic $\Pp\Pp$ collision cross section of 69.2\unit{mb}~\cite{Aaboud:2016mmw}.

\begin{table}[htb]\footnotesize
  \begin{center}
    \topcaption{List of \mc samples used. The cross sections $\sigma$ and number of events generated are also given. The \texttt{Herwig++} bulk graviton samples are used for assessing systematic uncertainty only. The cross sections are from McM and at LO. These are not used for the theoretical prediction, for which, the numbers in Table~\ref{CrossSectionsTable} is used.\label{tab:signal_MC}}
    \begin{tabular}{l|c|c}
      \hline
      \hline
      \multicolumn{3}{c}{Resonant Bulk graviton} \\ \cline{1-3}
      Process & $\sigma$ (pb) (LO) & Events\\ \hline
      \hline
      {GluGluToBulkGravitonToHHTo4B\_M-500\_narrow\_13TeV-madgraph} & 123. & 100000 \\
      {GluGluToBulkGravitonToHHTo4B\_M-550\_narrow\_13TeV-madgraph} & 84. & 100000 \\
      {GluGluToBulkGravitonToHHTo4B\_M-600\_narrow\_13TeV-madgraph} & 57.3 & 100000 \\
      {GluGluToBulkGravitonToHHTo4B\_M-650\_narrow\_13TeV-madgraph} & 35.6 & 100000 \\
      {GluGluToBulkGravitonToHHTo4B\_M-750\_narrow\_13TeV-madgraph} & 20.4 & 100000 \\
      {GluGluToBulkGravitonToHHTo4B\_M-800\_narrow\_13TeV-madgraph} & 14.0 & 100000 \\
      {GluGluToBulkGravitonToHHTo4B\_M-900\_narrow\_13TeV-madgraph} & 8.16 & 100000 \\
      {BulkGravTohhTohbbhbb\_narrow\_M-1000\_13TeV-madgraph} & 4.74 &50000 \\
      {BulkGravTohhTohbbhbb\_narrow\_M-1200\_13TeV-madgraph} & 1.90 & 50000 \\
      {BulkGravTohhTohbbhbb\_narrow\_M-1400\_13TeV-madgraph} & 0.763 & 50000 \\
      {BulkGravTohhTohbbhbb\_narrow\_M-1600\_13TeV-madgraph} & 0.33 & 50000 \\
      {BulkGravTohhTohbbhbb\_narrow\_M-1800\_13TeV-madgraph} & 0.155 & 48400 \\
      {BulkGravTohhTohbbhbb\_narrow\_M-2000\_13TeV-madgraph} & 0.0765 & 50000 \\
      {BulkGravTohhTohbbhbb\_narrow\_M-2500\_13TeV-madgraph} & 0.00158 & 50000 \\
      {BulkGravTohhTohbbhbb\_narrow\_M-3000\_13TeV-madgraph} & 0.000373 & 50000 \\
      \hline
      \multicolumn{3}{c}{Bulk graviton \texttt{Herwig++ samples}} \\ \cline{1-3}
      {BulkGravTohhTohbbhbb\_narrow\_M-1000\_13TeV-madgraph-herwig} & 4.74  & 50000 \\
      \hline
      \multicolumn{3}{c}{Radion} \\ \cline{1-3}
      \hline
      {GluGluToRadionToHHTo4B\_M-600\_narrow\_13TeV-madgraph} & 101. & 50000 \\
      {GluGluToRadionToHHTo4B\_M-650\_narrow\_13TeV-madgraph} & 79.2 & 50000 \\
      {GluGluToRadionToHHTo4B\_M-750\_narrow\_13TeV-madgraph} & 52.7 & 49800 \\
      {GluGluToRadionToHHTo4B\_M-800\_narrow\_13TeV-madgraph} & 43.7 & 100000 \\
      {RadionTohhTohbbhbb\_narrow\_M-1000\_13TeV-madgraph} & 21.1  & 50000 \\
      {RadionTohhTohbbhbb\_narrow\_M-1200\_13TeV-madgraph} & 11.2 & 50000 \\
      {RadionTohhTohbbhbb\_narrow\_M-1400\_13TeV-madgraph} & 5.99 & 50000 \\
      {RadionTohhTohbbhbb\_narrow\_M-1600\_13TeV-madgraph} & 3.33 & 50000 \\
 %     \hline
 %     \multicolumn{3}{c}{Non-resonant Bulk graviton} \\ \cline{1-3}
 %     \hline
 %         {GluGluToHHTo4B\_node\_10\_13TeV-madgraph} & -- & 300000 \\
 %         {GluGluToHHTo4B\_node\_11\_13TeV-madgraph} & -- & 300000 \\
 %         {GluGluToHHTo4B\_node\_12\_13TeV-madgraph} & -- & 300000 \\
 %         {GluGluToHHTo4B\_node\_13\_13TeV-madgraph} & -- & 300000 \\
 %         {GluGluToHHTo4B\_node\_2\_13TeV-madgraph} & -- & 300000 \\
 %         {GluGluToHHTo4B\_node\_3\_13TeV-madgraph} & -- & 300000 \\
 %         {GluGluToHHTo4B\_node\_4\_13TeV-madgraph} & -- & 300000 \\
 %         {GluGluToHHTo4B\_node\_5\_13TeV-madgraph} & -- & 300000 \\
%          {GluGluToHHTo4B\_node\_6\_13TeV-madgraph} & -- & 300000 \\
%          {GluGluToHHTo4B\_node\_7\_13TeV-madgraph} & -- & 300000 \\
%          {GluGluToHHTo4B\_node\_8\_13TeV-madgraph} & -- & 300000 \\
%          {GluGluToHHTo4B\_node\_9\_13TeV-madgraph} & -- & 300000 \\
%          {GluGluToHHTo4B\_node\_SM\_13TeV-madgraph}  & -- & 300000 \\
%          {GluGluToHHTo4B\_node\_box\_13TeV-madgraph} & -- & 300000 \\
      \hline
      \hline
    \end{tabular}
  \end{center}
\end{table}

\begin{table}[htb]\footnotesize
  \begin{center}
    \topcaption{List of background Monte Carlo samples used. The two \ttjets \textsc{Powheg} samples correspond to two different productions with the same generator parameters, but the latter with a much higher statistics. The cross sections $\sigma$ and number of events generated are also given. The cross sections for the QCD processes are at LO and are taken from McM. The other SM background cross sections are taken from Ref.~\cite{SMXsecTWiki13TeV}.\label{tab:bkg_MC}}
    \begin{tabular}{l|c|c}
      \hline
      \hline
      \multicolumn{3}{c}{Background} \\ \cline{1-3}
      Process & $\sigma$ (pb) & size \\
      \hline
      {QCD\_HT-100to200}   & $2.785\times 10^7 $ (LO) & 81,906,377 \\
      {QCD\_HT-200to300}   & $1.717\times 10^6 $ (LO) & 18,752,566 \\
      {QCD\_HT-300to500}   & $3.513\times 10^5 $ (LO) & 20,312,907 \\
      {QCD\_HT-500to700}   & $3.163\times 10^4 $ (LO) & 19,755,616 \\
      {QCD\_HT-700to1000}  & $6.831\times 10^3$  (LO) & 15,595,234 \\
      {QCD\_HT-1000to1500} & $1.207\times 10^3$  (LO) & 4,966,123  \\
      {QCD\_HT-1500to2000} & $119.9 $            (LO) & 3,964,488  \\
      {QCD\_HT-2000toinf}  & $25.24 $            (LO) & 1,984,407  \\
      \hline
      {TT\_TuneCUETP8M1\_13TeV-powheg-pythia8} & $831.76$ (NNLO) & 19,757,190 \\
      {TT\_TuneCUETP8M1\_13TeV-powheg-pythia8} & $831.76$ (NNLO) & 96,834,559 \\
      \hline
      \hline
    \end{tabular}
  \end{center}
\end{table}

\subsection{Theory predictions\label{ss:SignalXsec}}

\begin{table}[h]
  \begin{center}
    \topcaption{\small NLO Bulk graviton and radion cross sections cross sections times the decay rate for the samples that are used to set a limit (lower mass samples were studied but the analysis is not sensitive below 700 GeV). The bulk graviton production cross sections are evaluated for $k/\overline{\Mpl} = 0.1$~\cite{WED_BG_13TeV}, and the radion cross sections for $\Lambda_{\rm R} = 3\TeV$ and $kl = 35$~\cite{WED_radion_13TeV}. We note that for $k/\overline{\Mpl} = 0.1$, the decay width is not entirely negligible compared to the experimental resolution. The bulk graviton and radion to $\PH\PH$ decay branching fractions are taken from Refs.~\cite{WED_BGHHDecay_13TeV} and \cite{WED_radionHHDecay_13TeV}, respectively. All cross sections listed in Ref.~\cite{KKGraviton_Bulk_github}.\label{CrossSectionsTable}}
    \begin{tabular}{l|ccccc} 
      \hline
      \hline
      $\mx$ (GeV) & \multicolumn{2}{c}{Bulk graviton (fb)} & \multicolumn{2}{c}{Radion (fb)} \\ \cline{2-5}
      & $\sigma(\Pp\Pp \to X)$ & $\sigma(\Pp\Pp \to X \to \HH )$ & $\sigma(\Pp\Pp \to X)$ & $\sigma(\Pp\Pp \to X \to \HH )$ \\
      \hline
%      500   &   --      &   --   \\
%      550   &   --      &   --   \\
%      600   &   --      &   --   \\
%      650   &   --      &   --   \\
      750   & 24.916    & 2.408   & 655.75 & 155.46 \\
      800   & 18.199    & 1.771  & 543.45 & 128.68 \\
      900   & 9.802     & 0.963    & 373.56 & 88.433 \\
      1000  & 5.666     & 0.559   & 261.89 & 62.057 \\
      1500  & 0.573     & 0.057    & 54.428 & 12.897 \\
      1800  & 0.183     & 0.018    & 23.913 & 5.6664 \\
      2000  & 9.06E-02  & 9.03E-03  & 14.293 & 3.3868 \\
      %2500  & 1.87E-02  & 1.86E-03 \\ %& 4.3017 & 1.0193 \\
      %3000  & 4.41E-03  & 4.40E-04 \\ %& 1.3843 & 0.3280 \\
      %3500  & 1.15E-03  & 1.15E-04 \\ %& 0.4700 & 0.1114 \\
      %4500  & 8.93E-05  & 8.91E-06 \\ %& 5.34E-02 & 1.26E-02 \\
      \hline
      \hline
    \end{tabular}
  \end{center}
\end{table}

The NLO bulk graviton and radion cross sections are provided in Table~\ref{CrossSectionsTable} for the mass points used in the analysis. Lower mass points were considered but the analysis is not sensitive below 700 GeV. The bulk graviton production cross sections are evaluated for $k/\overline{\Mpl} = 0.1$~\cite{WED_BG_13TeV}, and the radion cross sections for $\Lambda_{\rm R} = 3\TeV$ and $kl = 35$~\cite{WED_radion_13TeV}. The bulk graviton and radion to $\PH\PH$ decay branching fractions are taken from Refs.~\cite{WED_BGHHDecay_13TeV} and \cite{WED_radionHHDecay_13TeV}, respectively.


