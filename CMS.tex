\chapter{Searching for New Particles}\label{Sec:CMS}

One of the most effective ways to probe the SM further and have a chance at discovering new BSM particles is to smash SM particles together at high energies in hopes of discovering something new. One of the easiest colliders to imagine is an electron-positron collider. Many have existed over time and account for many of the precision SM measurements. High energy collisions allow for a greater chance of something new and massive appearing, and allow for more precise measurements. However, it is hard to produce high energy collisions with electrons and positrons. Charged particles radiate energy when they are traveling through an electromagnetic field, where the power radiated is given by
\begin{equation}\label{Eq:SynchRad}
   P = \frac{e^4}{6\pi m^4c^5}E^2B^2
\end{equation}
Due to the low mass of electrons and positrons, they radiate much more energy than a heavier particle would. In order to reach high energies, then, colliding protons provides the most effective solution to date, and this is exactly what the Large Hadron Collider at CERN was built to do.
\vspace{5mm}
%For example, a positron might be accelerate by the Sun's tumultuous electromagnetic fields to very high energy and it might collide with an atom in our atmosphere. The positron might annihilate with one of the electrons in that atom (through the process in Figure \ref{Fig:Intro:Vertex1}). Even if the electron was at rest, the positron carried sufficient energy that the resulting photon could decay into something heavier than the electron: a pair of muons for example. This is exactly how muons are produced in the atmosphere, and in fact, at sea level every square meter is showered with about a hundred muons per second.

%Other particles should be accessible this way, but as can be seen in Figure \ref{Fig:Detect:Rates}, their rates are considerably lower than the muon. We would need to wait a very long time to have enough $W$s pass through our detector to have any hope of making a reasonable measurment.
%\begin{figure}[h!]
  %  \centering
 %       \includegraphics[width=\textwidth]{F3/atmodepth}
 %       \caption{Fluxes of common particles created in the atmosphere by cosmic rays. Note that muons are the most common charged particles produced in these interactions.}
%        \label{Fig:Detect:Rates}
%\end{figure}
%The muon is relatively long-lived, while the bosons and heavier quarks would decay in flight. The muon only makes it to our detectors because it is moving near the speed of light and benefits from the effects of time dilation\textbf{The faster something goes, the slower time runs for it}.

%Clearly then we need to force positron and electrons to collide on our terms, preferably at very high energies.

%The most basic collider is a long, straight tunnel which fires a beam of electrons at a beam of positrons. Most of the positron and electrons will pass by each other, with only a few actually interacting. Because of this, it makes more sense to build circular rings spinning the particles in opposite directions. This way a bunch\textbf{this is a technical term, as we'll discuss later} of electrons isn't wasted if it doesn't produce any interesting interactions with the positrons: it'll whip around the ring for a second pass, and a third, until it's depleted.

%The \textit{Large Electron-Positron} Collider at CERN was precisely such a machine, and was able to produce collisions with a center of mass energy of 206 GeV. More than enough energy to produce Z and W bosons (then the heaviest particles around). There are however some drawbacks to electron-positron colliders: charged particles radiate energy when they are bent by an electro-magnetic field, according to the following formula:
%\begin{equation}\label{Eq:SynchRad}
 %   P = \frac{e^4}{6\pi m^4c^5}E^2B^2
%\end{equation}
%For the E (electric) and B (magnetic) fields at the LEP collider, this was equivalent to $P \approx 0.2$mW \textit{per electron}. This may not seem like much, but even before LEP was opperating at its full potential the total synchrotron radiation output measured around thirteen megawatts\textbf{For reference, a modern locomotive produced about 6MW of power}. Such losses can quickly become overwhelming, especially if we plan on detecting particles several orders of magnitude heavier than the Z boson.

%Fortunately, the radiated power depends not just on the E and B fields, but on the mass of the particles being accelerated: from Equation \ref{Eq:SynchRad} we can see that this dependence is $\sim 1/m^4$. If we accelerated protons instead of electrons, we decrease the radiated power by a factor of $10^{13}$.

\section{The Large Hadron Collider}

The Large Hadron Collider (LHC)~\cite{lhcbrochure} is the largest collider ever built, colliding protons at the highest energy produced inside a collider, and producing more data than all other experiments combined. The LHC is located in a tunnel 100 m underground on the border of Switzerland and France, near Geneva, Switzerland. It has been running since 2007, colliding protons at a center of mass energy 7 TeV from 2007-2011, 8 TeV from 2011-2015, and now at 13 TeV. 

In order to create the proton beams used for collision in the LHC, the energy of the protons used for collisions is ramped up slowly. Hydrogen is stripped of electrons and fed into a linear accelerator, and then a succession of increasingly large circular synchrotrons. A schematic of the LHC can be found in Figure \ref{Fig:Detect:LHC}.
\begin{figure}[h!]
    \centering
        \includegraphics[width=0.49\textwidth]{F3/LHC}
        \includegraphics[width=0.49\textwidth]{F3/LHC2}
        \caption{(left) Schematic of the LHC and the smaller accelerators which feed into the LHC. (right) Aerial view of Geneva. The larger circle marks the location of the LHC, safely underneath the ground.}
        \label{Fig:Detect:LHC}
\end{figure}
Once the two beams of protons enter the LHC, they are accelerated to collision energies (currently 13 TeV, or 6.5 TeV per beam). These beams travel in opposite directions around the ring, which contains 1,232 dipole magnets to direct and accelerate the beams and 392 quadrupole magnets to focus the beams. The superconducting magnets are cooled to less than 2 K and produce a 7 T field. The protons travel in bunches of roughly 115 billlion each, with a spacing of 25 nanoseconds between crossings, barreling along at 99.99999999$\%$ the speed of light to collide within four detectors: ALICE, ATLAS, CMS, and LHCb.

% The LHC requires almost 100 tonnes of superfuild liquid helium to remain operational.
%\begin{figure}[h!]
  %  \centering
     %   \includegraphics[width=\textwidth]{F3/quadpole}
        %\caption{Quadrupole magnetic field, with the direction of the force felt by a charged particle in the field shown in blue. Alternating the N and S faces of the magnets allows the fields to focus a particle beam in both the vertical and horizontal directions.}
        %\label{Fig:Detect:QuadPole}
%\end{figure}
%At its current collision energy, the protons reach a top speed of 99.99999999$\%$ the speed of light\textbf{which is significantly faster than a Ferrari}. The beams are not continuous, but rather are composed of \textit{bunches} of protons, each bunch containing roughly 115 billion protons.
%The beams are made to cross at four experiments along the ring (see Figure \ref{Fig:Detect:LHC}. The spacing between bunches leaves 25 nanoseconds between crossings, equivalent to a collision rate of 40 MHz.
\vspace{5mm}

\subsection{A \textbf{proton-proton} Collision}

Because protons are composite particles, their collisions are not simple. While we speak of protons as being comprised of two up quarks and one down quark, these are in reality the valence quarks of the proton. The inner-workings of a proton are actually a complication, ongoing interaction between these three valence quarks, where this interaction involves gluons. These gluons also interact with each other, spontaneously producing different quarks which interact to become gluons again. Therefore, when protons are collided at the energy scales of a machine such as the LHC, we are actually colliding quarks and gluons, rather than protons. We call these quarks and gluons partons.

%Alice start here (first paragraph is yours)

This makes it much more difficult to know what has actually been collided each collision. We rely on Parton Distribution Functions~\cite{Bourilkov:2006cj,Ball:2014uwa} for a statistical understanding of what may be happening. Figure \ref{Fig:Detect:PDFset} shows two different simulated PDF sets. The x-axis represents the fraction of energy of the proton that belongs to a particular parton, while the y-axis tells us the probability of a particular parton having this particular energy. The various curves represent different partons, where the subscript "v" indicates a valence quark. The left plot shows a low energy scale, while the right plot is for the LHC scale. For example, at the LHC scale, there is a $\sim50$\% probability that an up valence quark is carrying 10\% of the proton's energy, or 0.65 TeV. 
\begin{figure}[h!]
    \centering
        \includegraphics[width=\textwidth]{F3/PDFset}
        \caption{PDF for 10 GeV (left) and 10 TeV (right) collisions. Each curve represents a parton in a proton, and tells us the probability (y-axis) of finding that parton carrying momentum fraction $x$ of the total momentum of the proton.}
        \label{Fig:Detect:PDFset}
\end{figure}
It is of interest to note that at higher energies, you are more likely to see gluon-gluon or quark-gluon interactions than you are at lower energies. It is important to have correct PDFs to predict the amount of new physics one might see at the LHC. For example, the non-resonant production considered in this thesis is only through gluon-gluon interactions. Unfortunately, it is hard to compute PDFs, so in order to have the best possible idea of what PDFs look like, partial models are combine with many measurements performed at fixed target and collider experiments to get more accurate values. There is a systematic uncertainty associated with this process which will be discussed in more detail in Section.

\vspace{5mm}
\section{The Compact Muon Solenoid}

\begin{figure}[h!]
    \centering
        \includegraphics[width=\textwidth]{F3/cms_0}
        \caption{The Compact Muon Solenoid detector.}
        \label{Fig:CMS:cms}
\end{figure}
In order to make use of these proton-proton collisions, a carefully designed detector is necessary to capture the data. There are two general purpose detectors on the LHC, which serve to look for new physics and test current SM predictions. The analysis presented in this thesis was performed on data taken with one of these two detectors: the Compact Muon Solenoid (CMS)\cite{Bayatian:922757}, as can be seen in Figure~\ref{Fig:CMS:cms}. Armed with a team of roughly 4,000 scientists, this five-story, 14,000 ton piece of hardware has been the subject of many previous theses and papers. A brief overview is presented here so as to give context to the data analysis performed in this thesis. A slice of the detector is sketched in Figure \ref{Fig:CMS:Slice}; each component will be described in the following sections to build an understanding of the detector as a whole.
\begin{figure}[h!]
    \centering
        \includegraphics[width=\textwidth]{F3/CMSnc}
        \includegraphics[width=\textwidth]{F3/cms_slice}
        \caption{Slice of the CMS detector, depicting the different components and the signatures of some of the different particles that pass through the detector after a proton-proton collision occurs.}
        \label{Fig:CMS:Slice}
\end{figure}

\vspace{5mm}
\subsection{Coordinates}

The CMS detector geometry approximates that of a large cylinder, with detectors in the barrel of the cylinder and at either end (endcaps). Coordinates in the detector are defined as shown in Figure~\ref{Fig:CMS:coordinates}. 
\begin{figure}[h!]
    \centering
    \includegraphics[width=0.49\textwidth]{F3/img_cms_coordinates.png}
    \caption{Coordinates of the CMS detector.}
    \label{Fig:CMS:coordinates}
\end{figure} 
The z-axis is parallel to the beam line - protons come in from the +z-axis and -z-axis. The y-axis points directory up and the x-axis points directly sideways. The azimuthal angle $\phi$ is in reference to the x-axis, where $\phi=0$ is along the x-axis. The polar angle $\theta$ is parametrized in terms of pseudorapidity $\eta$, where 
\begin{equation}
    \eta = -\ln\bigg{(}\tan\frac{\theta}{2}\bigg{)}
\end{equation}
We use $\eta$ instead of $\theta$ because pseudorapidity is a Lorentz-invariant quantity, meaning that it is the same in any Lorentz frame. This is critical for any quantity measured with respect to the beam line, since partons may not be symmetric in the laboratory rest frame because they may carry different fractions of energy of their respective protons.

Three quantities are recorded for each particle passing through the CMS detector: azimuthal angle $\phi$, pseudorapidity $\eta$, and transverse momentum $p_{T}$, defined as $p_{T} = \sqrt{p_{X}^{2}+p_{Y}^{2}}$, or the momentum in the plane transverse to the beam line. Transverse momentum is used rather than momentum because the initial state of collisions has $p_{T}=0$, while $p_{Z}$ is hard to determine. Momentum in Cartesian coordinates can be calculated with the following transformations:
\begin{equation}
p_X = p_T\cos\phi, \ \ \ \ 
p_Y = p_T\sin\phi, \ \ \ \
p_Z = p_T\cosh\eta
\end{equation}
\vspace{5mm}

\subsection{The Tracker}

The innermost layer of the detector is a tracker made out of silicon. The tracker records the paths of charged particles as they pass through. It has 13 layers in the barrel, and 14 in the endcaps. The first four layers are made up of silicon pixels, while the remaining layers are strips. Each pixel is 100$\times$150$\mu$m in area, while each strip is 180$\mu$m by 10 cm or 25 cm, depending on where the strip is located. While this adds up to over 200 $\text{m}^{2}$ of silicon sensors, the detector itself is not much larger than a shoe box. 

Charged particles passing through the detector interact with the silicon to produce "hits" which allows us to determine a particle's location to within 10$\mu$m. Because the tracker is inside a strong magnetic field (described below), the tracks of these charged particles bend. The higher momentum the particle, the less curved the track; this allows us to make a measurement on the charge and momentum of the particle.
\vspace{5mm}

\subsection{The Calorimeters}

The electromagnetic calorimeter is next to the tracker, followed by the hadronic calorimeter. While the purpose of a tracker is to allow a particle to pass through the detector in order to record the full track of a particle, the purpose of a calorimeter is to stop a particle so that a full measurement of the particle's energy can be obtained. Unlike the tracker, both charged and neutral particles interact with the calorimeter.

The electromagnetic calorimeter (ECAL) is built to measure the energy deposited by electrons, positrons, and photons. It is made of $\sim$80,000 lead-tungstate ($PbWO_4$) crystals. Lead-tungstate crystal emits light, or scintillates, when particles deposit energy in the compound. It was chosen for its short radiation length and short Moliere length, both of which force particles to stop faster, allowing for a compact ECAL. By measuring the light produced by a particle in the ECAL, we can measure the energy of the particle. Particles heavier than an electron or photon continue on through the detector to be measured by the hadronic calorimeter.

The hadronic calorimeter (HCAL) is built to measure the energy deposited by hadrons. It is comprised of layers of an absorber, which causes an electromagnetic shower when a particle deposits energy, alternated with layers of an active medium, which measures the light emitted in the electromagnetic shower. The absorber chosen for the HCAL is brass, made out of Russian naval shells recycled from WW2, and chosen for its short interaction length (once again allowing for a more compact detector) and because its non-magnetic. The active medium is a plastic scintillator, which allows us to measure the amount of light caused by both charged and neutral hadron interactions with the brass, and therefore allows us to measure the energy of these particles. 

%The Electromagnetic Calorimeter (ECAL) is composed of almost 80,000 lead-tungstate ($PbWO_4$) crystals. These crystals are a type of scintillator: a material which emit light when a particle deposits energy in it. Lead-Tungstate scintillates when a light charged particle (especially electrons) impact them. This light is proportional to the energy of the particle, and is  collected to get a measure of the energy deposited. While photons have no charge, they can either pair produce electron-positron pairs, or directly interact with an electron in the crystal. For both electrons and photon processes, the initital radiation from the collision is typically energetic enough to interact several times in the crystal, producing a shower of electrons and light as byproducts of the initial interaction also cause scintillation. Heavier particles punch through these crystals leaving little energy. They are instead measured by the Hadronic Calorimeter.

%\begin{figure}[h!]
   % \centering
     %   \includegraphics[width=0.66\textwidth]{F3/hcal}
       % \caption{The Hadronic Calorimeter, composed of 600 tons of brass, mostly recycled from WW2 naval shells.}
        %\label{Fig:CMS:Hadronic}
%\end{figure}
\vspace{5mm}
\subsection{The Solenoid}

By choosing the material for the inner three components carefully, the CMS design allows for all three of these detectors to fit within a large solenoid magnet, which produces a 3.8 Tesla magnetic field. This field causes the tracks of charged particles to bend within the inner three detectors, allowing us to make momentum measurements from tracks in the tracker.
% The field is powerful enough to shift the alignment of the detector, an effect which must be properly accounted for. 
\vspace{5mm}

\subsection{The Muon Chambers}

The last and largest component of the detector is the muon chambers. It is designed to measure the location, momentum, and energy of muons, which are too heavy to be stopped by the ECAL but not heavy enough to be stopped by the HCAL. This detector is comprised of three different types of detectors (drift tube chambers, cathode strip chambers, resistive plate chambers), all of which operate on the same principle; as the muon travels through the detector, it knocks electrons off of gas atoms, which are then collected to measure the energy and location of the muon. These detectors are alternated with layers of steel, which stop non-muons from passing through. These layers also direct and contain the weak magnetic field outside of the solenoid, which allows for a precise momentum measurement of the muon through measuring its curvature both in the tracker and in the muon chambers (it curves opposite directions in each of these, as can been seen in Figure~\ref{Fig:CMS:Slice}.
\vspace{5mm}

\subsection{What's Missing?}

Some particles, such as neutrinos, are so weakly interacting that they pass right through the detector. We would potentially expect the same from some BSM particles, such as dark matter. However, the detector is designed to measure a particle going almost any direction. In the x and y direction, or the transverse plane, which is perpendicular to the beam line, the initial state condition requires that total momentum is 0, where all of the momentum is in the z-direction (Figure~\ref{prepostCollision}, top). Having measured all particles that are registered by the detector carefully after the collision, with detectors that leave very little space uncovered for a particle to slip through, we are then able to reconstruct whether there is any missing energy in the transverse plane based off of energy conservation laws (Figure~\ref{prepostCollision}, bottom). While we can't say anything definitive about what particle(s) is(are) missing, we can report the total missing transverse momentum $p_{T}$ and the azimuthal angle ($\phi$) associated with this missing energy.
\begin{figure}[h!]
\centering
\includegraphics[width=\textwidth]{F3/preCollision.png}\\
\includegraphics[width=\textwidth]{F3/postCollision.png}
\caption{Diagram of particles inside the detector pre-collision (top) and post-collision (bottom), where post-collision shows where the protons entered, which particles were recorded (blue), and the direction of the missing energy (orange).}
\label{prepostCollision}
\end{figure}

\vspace{5mm}
\subsection{The Particle Flow Algorithm}

Information from all of the detectors is collected and anlayzed by the Particle Flow Algorithm (PF Algorithm, or PF)~\cite{CMS-PAS-PFT-09-001}, which allows for an accurate event by event reconstruction by combining information from detectors, rather than treating the information separately.
\vspace{5mm} 

\subsection{Jets}

As mentioned in Chapter \ref{Sec:Intro}, all particles that exist in nature must be color-neutral. When proton-proton collision result in bare quarks or gluon, the process of hadronization begins immediately, and new quarks are produced from the vacuum until it is no longer energetically favorable. This creates a spray of hadronic activity; wherever one quark or gluon is, many more will exist in a conical structure that points in the direction the original quark or gluon was moving. A depiction of these in a real event at the LHC is shown in Figure~\ref{Fig:CMS:Jet}, where the yellow cones represent these hadronic showers, which are called jets.
\begin{figure}[h!]
    \centering
        \includegraphics[width=\textwidth]{F3/Jets}
        \caption{Typical CMS event with large number of hadrons (tracks represented by green lines, HCAL energy in blue, ECAL energy in red). These hadrons are collected into \textit{jets} (shown by yellow triangles). The PF algorithm combines information from the trackers and calorimeters to properly measure the total momentum and energy of the jet.}
        \label{Fig:CMS:Jet}
\end{figure}
Jets are composed of constituents which are defined by combining tracker, calorimeter, and muon system information from the PF Algorithm. These constituents are clustered into jets, making educated as to which constituents belong to which jets. A number of algorithms exist for clustering, but the jets used in this analysis are clustered with the anti-$K_{T}$ algorithm~\cite{Cacciari:2008gp}. Using this method, every PF candidate, or particle reconstructed with the PF Algorithm, is compared with all of the other candidates, measuring a distance-like parameter between each pair. The two closest constituents are paired to become a new constituent. This process continues until the distance-like parameter between the jet and the beam is equal to $1/p_T^2$ of this new conglomerate constituent. Then this constituent is considered a jet and is removed from consideration. This continues until no constituents remain.

The anti-$K_{T}$ algorithm tends to produce conical jets with smooth, rounded edges. Jets of this type are referred to as AK$R$ jets, where $R$ is the radius of the jet. Two types of jet are used in this analysis: AK8 and AK4, with R=0.8 and R=0.4 respectively, depending on how collimated we expect decay products to be.

%Every \textit{PF Candidate} (individual reconstructed particles) is considered against every other candidate and the distance-like parameter:
%\begin{equation}
%   d_{ij} = \textrm{min}(p_{T,i}^{-2},p_{T,i}^{-2})\frac{(y_i-y_j)^2+(\phi_i - \phi_2)^2}{R^2}
%\end{equation}
%is measured. Here $R$ is a distance parameter which sets the size of the jet. The two closest constituents are paired together and become a new constituent. This pairing continues, adding new constituents to this and other \textit{pseudo-jets} until $d_iB$ (where $B$ is the beam) is equal to $1/p_T^2$ of a pseudo-jet. When this is the case, that pseudo-jet is classified as a jet and it is removed from consideration. The algorithm continues for remaining constituents and pseudo-jets until no constituents remain.

%The Anti-$K_{T}$ algorithm is notable for creating ``conical'' jets with smooth, rounded edges (see again Figure \ref{Fig:CMS:JetAlgo}) although this is not an exact statement, especially when two jets are near each other. We will refer to jets created with this algorithm as AK$R$ jets, where $R$ is the distance parameter used. We use two different kinds of jets in our analysis: AK8 and AK4, with R respectively 0.8 and 0.4. While it is impossible to reconstruct the small masses of most quarks, the energies and momentums of the jet can be used to measure their kinematic properties. More massive particles such as the Higgs or W bosons and the very heavy top quark yield jets whose mass can be measured with some degree of accuracy. We use AK8 jets to reconstruct heavier objects.
\vspace{5mm}
%ALICE START HERE
\subsection{Triggers}
The detector output for each bunch-crossing would take approximately a megabyte to store. The crossing rate is forty megahertz, a rate well above what any modern computing system can handle. Because of this, the majority of collisions at CMS are discarded. The \textit{trigger system} is responsible for pruning this output of uninteresting events.

The trigger operates in two main stages: the \textit{Level 1} trigger is a hardware trigger. Detector output is stored in a buffer and analyzed by custom built FPGA circuits. This analysis looks for key markers of ``interesting'' physics such as a high energy muon, or very large deposits of energy in the HCAL or ECAL. This stage allows us to reject all but about $0.1\%$ of collisions. The buffered data is released to the next stage, at a rate of around fifty kilohertz.

The \textit{High Level} trigger takes the output of the Level 1 trigger and analyzes it further, separating interesting events for further study and rejecting all others. A large number of triggers are available, and there are further sorted into \textit{Datasets} sharing a common characteristic. For example, the \textit{SingleMuon} dataset consists of a large logical \textit{or} of possible \textit{ High Level Trigger Paths}. Example of such paths include \textit{HLT\_Mu50} (an event with a 50 GeV muon), \textit{HLT\_Mu45Eta2p1} (an event with a 45 GeV muon in with $|\eta| < 2.1$) and others. Most triggers used in analysis attempt to keep all possible events, but \textit{prescale} Trigger are also kept, with only a fraction of triggering events allowed through. These triggers are used to measure the efficiency of the un-prescaled analysis triggers.

The resulting rate is about one kilohertz. There are the events which Physics Analyses are performed on. Our search for $Z^\prime$ will use two triggers: the \textit{Mu45\_eta2p1} trigger which collects events with one high energy (45 GeV) muon in the barrel of the detector, and the \textit{El45\_PFJet200\_PFJet50} trigger, which collects events with one high energy electron (45 GeV) and at least two jets, one with energy at least 200 GeV and the other with energy at least 50 GeV.

Modeling the trigger is a crucial component of good simulations of the detector, especially when modeling new physics signals. Uncertainties in the overall efficiency of these triggers is an important systematics uncertainty of the analysis and will be further discussed.

\subsection{Pileup}
An \textit{event} in CMS, ideally, is the outcome of a particular proton-proton collision. Most events do not yield any new physics or even ``interesting'' physics and the goal of our analysis will be to reduce the number of events in consideration to as small a set as possible.

Unfortunately, this ideal picture is not representative of the actual output of the detector because in each bunch-crossing there can be multiple proton-proton interactions. An event may therefore consist of up to forty individual interactions. It is therefore necessary to define a primary vertex: a \textit{vertex} is a point along the beam from which a large number of particle flow objects originate. The primary vertex is selected as the vertex with the highest value for the sum of the square of the transverse momenta of tracks and candidates associated with it.

Pileup can greatly affect the jet algorithms as particles produced in a vertex separate from the one we consider may be clustered into jets from the primary vertex. These \textit{pileup contributions} smear out the measurements of jet mass and momentum and must be accounted for in the analysis. Specialized variables described in the next section can reduce these effects and better identify the mass of a jet. The choice of the AK4 and AK8 jet clustering algorithms is in part predicated on the resistance of that algorithm to effects from \textit{soft} constituents (i.e. constituents from other vertices). This effect is shown in Figure \ref{Fig:CMS:NoPU}
\begin{figure}[h!]
    \centering
        \includegraphics[width=\textwidth]{F3/NoPU}
        \caption{Resolution of R = 1.0 jets from different algorithms. The $Anti-K_T$ Algorithm performs best.}
        \label{Fig:CMS:NoPU}
\end{figure}